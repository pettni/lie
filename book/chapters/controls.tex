% !TEX root = ../manuscript.tex


\chapter{Application: Geometric Control}

\begin{itemize_outcomes}
  \item Extend PD theory to Lie Groups.
  \item Quadrotor control on $SE(3)$
\end{itemize_outcomes}


In the following we let $M = SO(3)$ be the Lie group consisting of rotation matrices with matrix multiplication being the group action. We also write $e = I_3$ for the identity element of the group.

Consider the rigid body dynamics
\begin{subequations}
  \label{eq:attitude}
  \begin{align}
    \label{eq:attitude1} \dot R                 & = R \; {^R\hat \bomega},              \\
    \label{eq:attitude2} J \; {^R\dot{\bomega}} & = -{^R\hat\bomega} J {^R\bomega} + u,
  \end{align}
\end{subequations}
where $J$ is the moment of inertia, ${^R \hat \bomega} \in T M_R$ is the angular velocity in the body frame, and $R \in SO(3)$ is the attitude. By \eqref{eq:velocity_at_identity} we can see that the angular velocity ${^{e} \hat\bomega} \in T M_{e}$ in the inertial frame can be obtained as
\begin{equation}
  ^{e} \bomega = \Ad_R \left( ^R\bomega \right) = R {^R\bomega}.
\end{equation}
Using \eqref{eq:ad_relation1} we also see that $^e \hat\bomega = \widehat {\Ad_R ^R\bomega } = R  \; {^R\bomega} \; R^T$, so it follows that \eqref{eq:attitude1} can be written as
\begin{equation}
  \dot R = {^e\hat\bomega} R.
\end{equation}

We assume that a smooth trajectory in the inertial frame is given by $R_d$ and $^e \bomega_d$ satisfying the dynamics
\begin{equation}
  \label{eq:desired_dyn}
  \dot R_d = {^e \hat \bomega_d} R_d,
\end{equation}
and the goal is to control $u$ in \eqref{eq:attitude} so that $R$ and $^e\bomega$ are close to $R_d$ and $^e\bomega_d$.

\section{Error Functions}

In general we would like to pick for $\tilde e_r = R_d \ominus R$ the error function $\frac{1}{2}\| \tilde e_r \|^2$ with derivative $\left\langle \tilde e_r, \tilde e_\bomega \right\rangle$ for
\begin{equation}
  \tilde e_\bomega = {\dot {\tilde e}}_r = {J^{R_d \ominus R}_{R_d}} \; {^{R_d} \bomega_d} + {J^{R_d \ominus R}_{R}} \; {^{R} \bomega}.
\end{equation}
This is general for any Lie group, and we can pick $u$ to stabilize a double integrator system in the tangent space. However, the derivative of $\tilde e_\bomega$ is cumbersome to evaluate and it is possible to arrive at a simpler formulation in $SO(3)$. Consider the error functions
\begin{subequations}
  \begin{align}
    \Psi(R, R_d) & = 1 - \cos(\theta) = \frac{1 - \tr(R R_d^T)}{2} = - \frac{1 - \left\langle R_d, R \right\rangle_F}{2}, \\
    e_r          & = \frac{1}{2} (R_d^T R-R^T R_d)^\vee,                                                                  \\
    e_\bomega    & = \bomega - R^T \bomega^d \in T SO(3)_R.
  \end{align}
\end{subequations}
It can be seen by \eqref{eq:so3_log} that $e_r$ is a rescaling of $\tilde e_r$. The derivative of $\Psi$ is $\left\langle e_r, e_\bomega \right\rangle$ as above, indeed
\begin{equation}
  \begin{aligned}
    \dot \Psi = & -\frac{1}{2} \left( \left\langle R_d, \dot R \right\rangle_F + \left\langle \dot R_d, R \right\rangle_F \right) = - \frac{1}{2} \left( \left\langle R_d , R \hat \bomega \right\rangle_F + \left\langle \hat \bomega_d R_d , R \right\rangle_F \right) =                             \\
                & = - \frac{1}{2} \left( \left\langle R^T R_d , \hat \bomega \right\rangle_F - \left\langle \hat \bomega_d^T R_d , R \right\rangle_F \right) = - \frac{1}{2} \left( \left\langle R^T R_d , \hat \bomega \right\rangle_F - \left\langle  R_d , \hat \bomega_d R \right\rangle_F \right) \\
                & \overset{\eqref{eq:so3_adjoint_hat}}= - \frac{1}{2} \left( \left\langle R^T R_d , \hat \bomega \right\rangle_F - \left\langle  R_d , R \widehat {R^T\bomega_d} \right\rangle_F \right) = - \frac{1}{2} \left\langle R^T R_d , \hat e_\bomega \right\rangle_F                         \\
                & = \frac{1}{4} \left\langle R_d^T R - R^T R_d, \hat e_\bomega \right\rangle_F \overset{\eqref{eq:dot_trace_rel}}= e_r \cdot e_\bomega,
  \end{aligned}
\end{equation}
where we have used the property that the Frobenius product $\left\langle A, B \right\rangle_F = - \left\langle A^T, B \right\rangle_F$ for $B$ skew-symmetric.

\section{WIP: Derivative via Jacobians}

It should also be possible to calculate the time derivative via \eqref{eq:dtr_dr}, but it seems difficult to arrive at the same expression:
\begin{equation}
  \begin{aligned}
    \frac{\mathrm{d}}{\mathrm{d}t} \Psi = -\frac{1}{2} \frac{^e\partial \tr(R R_d^T)}{^e \partial R R_d^T} \left[ \frac{^e\partial R R_d^T}{^e \partial R} \; {^e\bomega} + \frac{^e\partial R R_d^T}{^e \partial R_d}  {^e\bomega_d}  \right] \\
    = -\frac{1}{2} \left( R_d R^T - R  R_d^T \right)^\vee \cdot \left[ {^e\bomega} - R {^e\bomega_d}  \right]
  \end{aligned}
\end{equation}
Using right derivatives instead:
\begin{equation}
  \begin{aligned}
    \frac{\mathrm{d}}{\mathrm{d}t} \Psi = -\frac{1}{2} \frac{^{\tr(R R_d^T)} \partial \tr(R R_d^T)}{^{R R_d^T} \partial R R_d^T} \left[ \frac{^{R R_d^T} \partial R R_d^T}{^R \partial R}  {^R\bomega} + \frac{^{R R_d^T}\partial R R_d^T}{^{R_d} \partial R_d} {^{R_d} \bomega_d}  \right] \\
    = -\frac{1}{2} \left( R_d R^T - R R_d^T \right)^\vee \left[ R_d \; {^R\bomega} - {^{R_d}\bomega_d}  \right]
  \end{aligned}
\end{equation}



\section{Lyapunov Stability}

We let the input be
\begin{equation}
  u = -k_r e_r - k_\bomega e_\bomega + \widehat{R^T \bomega_d} J R^T \bomega_d + J R^T \dot \bomega_d.
\end{equation}
and consider a Lyapunov candidate on the form
\begin{equation}
  V = \frac{1}{2} e_\bomega \cdot J e_\bomega + k_r \Psi + c e_r \cdot J e_\bomega
\end{equation}
The derivative of the Lyapunov candidate then \begin{proposition}
  It holds that
\end{proposition}
\begin{equation}
  \label{eq:myJeo_dot}
  J \dot e_\bomega = -k_r e_r - k_\bomega e_\bomega + \left( J e_\bomega + \left(2 J R^T \bomega_d - \text{trace}(J) I \right) R^T \bomega_d \right) \times e_\bomega.
\end{equation}
\begin{proof}
  \begin{equation*}
    \begin{aligned}
      \frac{\mathrm{d}}{\mathrm{d}t} J e_\bomega & \overset{\eqref{eq:e_omega}}= J \dot \bomega - J \dot R^T \bomega_d - J R^T \dot \bomega_d \overset{\eqref{eq:attitude}}= u - \hat \bomega J \bomega - J \left( R \hat \bomega \right)^T \bomega_d - J R^T \dot \bomega_d \\
                                                 & \overset{\eqref{eq:feedback}}= -k_r e_r - k_\bomega e_\bomega + \widehat{R^T \; {\bomega_d}} J R^T \; {\bomega_d} - \hat \bomega J \bomega - J \hat \bomega^T R^T \bomega_d                                               \\
                                                 & \overset{\eqref{eq:e_omega}, \eqref{eq:so3_transpose}}= -k_r e_r - k_\bomega e_\bomega + \widehat{R^T \; {\bomega_d}} J R^T \; {\bomega_d} - (\hat e_\bomega + \widehat{R^T \bomega_d}) J (e_\bomega + {R^T \bomega_d})   \\
                                                 & \quad + J \left( \hat e_\bomega + \cancelto{0}{\widehat{R^T \bomega_d}} \right) R^T \bomega_d                                                                                                                             \\
                                                 & \overset{\eqref{eq:hat_so_rel1}}= -k_r e_r - k_\bomega e_\bomega + \left( \widehat{J e_\bomega} + \widehat{J R^T \bomega_d} - \widehat{R^T \bomega_d} J - J \widehat{R^T \bomega_d} \right) e_\bomega                     \\
                                                 & \overset{\eqref{eq:hat_so_rel4}}= -k_r e_r - k_\bomega e_\bomega + \left( J e_\bomega + \left(2 J R^T \bomega_d - \text{trace}(J) I \right) R^T \bomega_d \right)^\wedge e_\bomega.
    \end{aligned}
  \end{equation*}
\end{proof}

We then get
\begin{equation}
  \begin{aligned}
    \dot V = -k_\bomega \| e_\bomega \|^2 + c \dot e_r \cdot J e_\bomega + c e_r \cdot J \dot e_\bomega
  \end{aligned}
\end{equation}
It remains to bound the terms involving $c$. From \eqref{eq:dot_trace_rel} we have that $\| \bomega \|_2^2 = \frac{1}{2} \| \hat \bomega \|_F^2$. We also have
\begin{equation}
  \frac{\mathrm{d}}{\mathrm{d}t} R_d^T R = R_d^T R \hat \bomega + R_d^T \hat \bomega_d^T R \overset{\eqref{eq:so3_transpose}} = R_d^T R \hat \bomega - R_d^T \hat \bomega_d R \overset{\eqref{eq:so3_adjoint_hat}}= R_d^T R \hat \bomega - R_d^T R \; \widehat{R^T \bomega_d} = R_d^T R \hat e_\bomega.
\end{equation}
and therefore we get that $\| \dot {\hat {e}}_r \|_F = \left\| \frac{1}{2} \left( R_d^T R \hat e_\bomega + \hat e_\bomega R^T R_d \right) \right\|_F \leq \| \hat e_\bomega \|_F$, so it follows that
\begin{equation}
  \| \dot e_r \|_2 \leq \| e_\bomega \|_2 \quad \implies \dot e_r \cdot J e_\bomega \leq \lambda_M(J) \| e_\bomega \|^2_2.
\end{equation}
Finally, using that $\| e_r \| \leq 1$,
\begin{equation}
  \begin{aligned}
    J \dot e_\bomega \cdot  e_r  \overset{\eqref{eq:myJeo_dot}}= \left( -k_r e_r - k_\bomega e_\bomega + \left( J e_\bomega + \left(2 J R^T \bomega_d - \text{trace}(J) I \right) R^T \bomega_d \right) \times e_\bomega \right) \cdot e_r \\
    \leq - k_r \| e_r \|^2 + k_\bomega \| e_r \| \| e_\bomega \| + \lambda_M(J) \| e_\bomega \|^2 + B \| e_\bomega \| \| e_r \|.
  \end{aligned}
\end{equation}
\begin{tcolorbox}
  We can now bound the derivative as follows:
  \begin{equation}
    \dot V \leq
    - \begin{bmatrix}
      \| e_r \| \\ \| e_\bomega \|
    \end{bmatrix}^T
    \begin{bmatrix}
      c k_r               & -c(k_\bomega + B)/2         \\
      -c(k_\bomega + B)/2 & k_\bomega - 2c \lambda_M(J)
    \end{bmatrix}
    \begin{bmatrix}
      \| e_r \| \\ \| e_\bomega \|
    \end{bmatrix},
  \end{equation}
  and it follows that if we choose $c$ small enough then the matrix is positive definite and thus $V$ decreases along trajectories of the closed-loop system.
\end{tcolorbox}


\section{Direction-driven Attitude Control on SO(3)}

We pick two orthogonal unit-length directions $b_1$ and $b_2$ and define the following error function:
\begin{equation}
  \Psi_i(R) = \frac{1}{2} \left\| R b_i - R_d b_i \right\|^2 = 1 - (R b_i) \cdot (R_d b_i).
\end{equation}
The derivative of $\Psi_i(R)$ becomes
\begin{equation}
  \label{eq:dot_psi}
  \begin{aligned}
    \dot \Psi_i(R) & = -\dot R b_i \cdot R_d b_i - R b_i \cdot \dot R_d b_i \overset{\eqref{eq:attitude1}, \eqref{eq:desired_dyn}}{=} -R {^R{\hat \bomega}} b_i \cdot R_d b_i - R b_i \cdot {^e\hat \bomega_d} R_d b_i                                                                                                                                   \\
                   & = - {^R{\hat \bomega}} b_i \cdot R^T R_d b_i - b_i \cdot R^T \; {^e\hat \bomega_d} R_d b_i \overset{\eqref{eq:so3_adjoint_hat}}{=} - {^R{\hat \bomega}} b_i \cdot R^T R_d b_i - b_i \cdot  \widehat{R^T \; {^e\bomega_d}} R^TR_d b_i                                                                                                \\
                   & \overset{\eqref{eq:hat_so_rel3}}{=} -{^R{\bomega}} \cdot \left( \widehat{b_i} \; R^T R_d b_i \right) - R^T \; {^e\bomega_d} \cdot \widehat{R^TR_d b_i} b_i \overset{\eqref{eq:hat_so_rel1}}{=} \underbrace{\left( {^R{ \bomega}} - R^T \; {^e\bomega_d} \right)}_{e_\bomega} \cdot \underbrace{\widehat{R^TR_d b_i} b_i}_{e_{r_i}},
  \end{aligned}
\end{equation}
where we have defined two error functions
\begin{subequations}
  \begin{align}
    \label{eq:e_ri} e_{r_i}      & = \widehat{R^TR_d b_i} b_i,           \\
    \label{eq:e_omega} e_\bomega & = {^R\bomega} - R^T \; {^e\bomega_d},
  \end{align}
\end{subequations}
that are small when $R \approx R_d$ and when $\Ad_R {^R{\hat \bomega}} = R {^R{\hat \bomega}} \approx {^e\bomega_d}$, respectively.

\section{Feedback Control}

\begin{tcolorbox}
  Given these error functions we consider the feedback control
  \begin{equation}
    \label{eq:feedback}
    u = -e_r - k_\bomega e_\bomega + \widehat{R^T \; {^e\bomega_d}} J R^T \; {^e\bomega_d} + J R^T {^e\dot \bomega_d},
  \end{equation}
  where
  \begin{equation}
    \label{eq:er}
    e_r = k_1 e_{r_1} + k_2 e_{r_2},
  \end{equation}
  and $k_1, k_2, k_\bomega$ are positive gains. Take the candidate Lyapunov function
  \begin{equation}
    V = \frac{1}{2} e_\bomega \cdot J e_\bomega + k_1 \Psi_1(R) + k_2 \Psi_2(R) + c J e_\bomega \cdot e_r.
  \end{equation}
\end{tcolorbox}

In the following we drop the upper left superscripts and write $\bomega = {^R\bomega}$ and $\bomega_d = {^e\bomega_d}$.


\section{Lyapunov lower bound}
We would like to show that $V = 0$ implies that $\| e_r \|$ and $\| e_\bomega \|$ are zero. The main challenge lies in bounding the terms containing $\Psi_i$. Note that
\begin{equation}
  \label{eq:eri_sin}
  \|e_{r_i}\| = \|\widehat{R^TR_d b_i} b_i\| = \|R^TR_d b_i \times b_i\| = \sin \theta_i,
\end{equation}
where $\theta_i$ is the angle between $R^TR_d b_i$ and $b_i$. Note that $\theta_i$ is always in the range $[0, \pi]$. Similarly,
\begin{equation}
  \label{eq:psi_cos}
  \Psi_i(R) = 1 - R^T R_d b_i \cdot b_i = 1 - \cos \theta_i.
\end{equation}
Utilizing this and $(a+b)^2 \leq 2(a^2 + b^2)$ we get:
\begin{equation*}
  \begin{aligned}
    \| e_r \|^2 & \overset{\eqref{eq:er}}= \| k_1 e_{r_1} + k_2 e_{r_2} \|^2 \leq ( k_1 \| e_{r_1} \| + k_2 \| e_{r_2} \| )^2 \overset{\eqref{eq:eri_sin}}= ( k_1 \sin \theta_1 + k_2 \sin \theta_2 )^2 \\
                & = \left( k_1 \sqrt{1 - \cos^2 \theta_1 } + k_2 \sqrt{1 - \cos^2 \theta_2 } \right)^2 \leq \left( k_1 \sqrt{2(1 - \cos \theta_1)} + k_2 \sqrt{2(1 - \cos \theta_2) } \right)^2         \\
                & \overset{\eqref{eq:psi_cos}}= 2 \left( k_1 \sqrt{\Psi_1(R)} + k_2 \sqrt{\Psi_2(R)} \right)^2 \leq 4 \min(k_1, k_2) \left( k_1\Psi_1(R) + k_2\Psi_2(R) \right).
  \end{aligned}
\end{equation*}
\begin{tcolorbox}
  We therefore get
  \begin{equation}
    V \geq \frac{1}{2} \begin{bmatrix}
      \| e_r \| \\ \| e_\bomega \|
    \end{bmatrix}^T
    \begin{bmatrix}
      \frac{1}{2\min(k_1, k_2)} & -c \lambda_M(J) \\
      -c \lambda_M(J)           & \lambda_m(J)
    \end{bmatrix}
    \begin{bmatrix}
      \| e_r \| \\ \| e_\bomega \|
    \end{bmatrix}
  \end{equation}
  where the matrix is positive definite for small enough $c$.
\end{tcolorbox}

\section{Lyapunov derivative}

We start with an intermediate result
\begin{proposition}
  It holds that
\end{proposition}
\begin{equation}
  \label{eq:Jeo_dot}
  J \dot e_\bomega = -e_r - k_\bomega e_\bomega + \left( J e_\bomega + \left(2 J R^T \bomega_d - \text{trace}(J) I \right) R^T \bomega_d \right) \times e_\bomega.
\end{equation}
\begin{proof}
  \begin{equation*}
    \begin{aligned}
      \frac{\mathrm{d}}{\mathrm{d}t} J e_\bomega & \overset{\eqref{eq:e_omega}}= J \dot \bomega - J \dot R^T \bomega_d - J R^T \dot \bomega_d \overset{\eqref{eq:attitude}}= u - \hat \bomega J \bomega - J \left( R \hat \bomega \right)^T \bomega_d - J R^T \dot \bomega_d \\
                                                 & \overset{\eqref{eq:feedback}}= -e_r - k_\bomega e_\bomega + \widehat{R^T \; {\bomega_d}} J R^T \; {\bomega_d} - \hat \bomega J \bomega - J \hat \bomega^T R^T \bomega_d                                                   \\
                                                 & \overset{\eqref{eq:e_omega}, \eqref{eq:so3_transpose}}= -e_r - k_\bomega e_\bomega + \widehat{R^T \; {\bomega_d}} J R^T \; {\bomega_d} - (\hat e_\bomega + \widehat{R^T \bomega_d}) J (e_\bomega + {R^T \bomega_d})       \\
                                                 & \quad + J \left( \hat e_\bomega + \cancelto{0}{\widehat{R^T \bomega_d}} \right) R^T \bomega_d                                                                                                                             \\
                                                 & \overset{\eqref{eq:hat_so_rel1}}= -e_r - k_\bomega e_\bomega + \left( \widehat{J e_\bomega} + \widehat{J R^T \bomega_d} - \widehat{R^T \bomega_d} J - J \widehat{R^T \bomega_d} \right) e_\bomega                         \\
                                                 & \overset{\eqref{eq:hat_so_rel4}}= -e_r - k_\bomega e_\bomega + \left( J e_\bomega + \left(2 J R^T \bomega_d - \text{trace}(J) I \right) R^T \bomega_d \right)^\wedge e_\bomega.
    \end{aligned}
  \end{equation*}
\end{proof}
Thus the derivative of $V$ is
\begin{equation}
  \begin{aligned}
    \dot V \overset{\eqref{eq:dot_psi}}= e_\bomega \cdot J \dot e_\bomega + e_r \cdot e_\bomega + c J \dot e_\bomega \cdot e_r + c J e_\bomega \cdot \dot e_r
    \overset{\eqref{eq:Jeo_dot}}= -k_\bomega \| e_\bomega \|^2 + c J \dot e_\bomega \cdot e_r + c J e_\bomega \cdot \dot e_r,
  \end{aligned}
\end{equation}
so we would like to bound $J \dot e_\bomega \cdot e_r$ and $J e_\bomega \cdot \dot e_r$ in terms of $\| e_\bomega \|$ and $\| e_r \|$.  First we have
\begin{equation}
  \label{eq:rdr_dt}
  \frac{\mathrm{d}}{\mathrm{d}t} R_d^T R = R_d^T R \hat \bomega + R_d^T \hat \bomega_d^T R \overset{\eqref{eq:so3_transpose}} = R_d^T R \hat \bomega - R_d^T \hat \bomega_d R \overset{\eqref{eq:so3_adjoint_hat}}= R_d^T R \hat \bomega - R_d^T R \; \widehat{R^T \bomega_d} = R_d^T R \hat e_\bomega.
\end{equation}
Now, $e_{r_i} = \widehat{R^T R_d b_i} b_i,$ so by linearity of the hat mapping and that $\| \hat b_i \| = \| b_i \| = 1$ it follows that
\begin{equation}
  \dot e_{r_i} = \widehat{R_d^T R \hat e_\bomega b_i} \; b_i = - \widehat{R_d^T R \hat b_i e_\bomega} \; b_i, \quad \implies \| \dot e_{r_i} \| \leq \| R_d^T R \| \| \hat b_i \| \| e_\bomega \| \| b_i \| = \| e_\bomega \|.
\end{equation}
Thus, for $\lambda_M(J)$ the maximal eigenvalue of $J$,
\begin{equation}
  \| J e_\bomega \cdot \dot e_r \| \leq \lambda_M(J) (k_1 + k_2) \| e_\bomega \|^2.
\end{equation}
Finally, we bound the last term, utilizing that $\| e_r \| \leq k_1 + k_2$:
\begin{equation}
  \begin{aligned}
    J \dot e_\bomega \cdot  e_r  \overset{\eqref{eq:Jeo_dot}}= \left( -e_r - k_\bomega e_\bomega + \left( J e_\bomega + \left(2 J R^T \bomega_d - \text{trace}(J) I \right) R^T \bomega_d \right) \times e_\bomega \right) \cdot e_r \\
    \leq - \| e_r \|^2 + k_\bomega \| e_r \| \| e_\bomega \| + \lambda_M(J) (k_1 + k_2) \| e_\bomega \|^2 + B \| e_\bomega \| \| e_r \|,
  \end{aligned}
\end{equation}
where $B$ is some number that upper bounds $\| \left( 2 J R^T \bomega_d - \text{trace}(J) I \right) R^T \bomega_d \|$.

\begin{tcolorbox}
  We can now bound the derivative as follows:
  \begin{equation}
    \dot V \leq
    - \begin{bmatrix}
      \| e_r \| \\ \| e_\bomega \|
    \end{bmatrix}^T
    \begin{bmatrix}
      c                   & -c(k_\bomega + B)/2                     \\
      -c(k_\bomega + B)/2 & k_\bomega - 2c \lambda_M(J) (k_1 + k_2)
    \end{bmatrix}
    \begin{bmatrix}
      \| e_r \| \\ \| e_\bomega \|
    \end{bmatrix},
  \end{equation}
  and it follows that if we choose $c$ small enough then the matrix is positive definite and thus $V$ decreases along trajectories of the closed-loop system.
\end{tcolorbox}

\textbf{Remaining steps:}

\begin{itemize}
  \item Show that undesired equilibria are unstable
\end{itemize}


\chapter{Application: Model-Predictive Control}

Consider a system $\X(t)$ evolving on a Matrix Lie group
\begin{equation}
  \label{eq:mpc_dynamics}
  \mathrm{d}^r \X_t = f(\X, u), \quad \X \in \M, \qquad f : \M \times U \rightarrow T \M.
\end{equation}
We are interested in finding an approximate solution to the optimal control problem
\begin{equation}
  \label{eq:mpc_ocp}
  \begin{cases}
    \min        & \int_0^T \left\| \sqrt{Q(\tau)} (\X(\tau) \ominus_r \X_d(\tau)) \right\|_2^2 + \left\| \sqrt{R(\tau)} (u(\tau) - u_d(\tau)) \right\| \mathrm{d} \tau + \left\| \sqrt{Q(T)} (\X(T) \ominus_r x_d(T)) \right\|_2^2 \\
    \text{s.t.} & \eqref{eq:mpc_dynamics}                                                                                                                                                                                          \\
                & \X(0) = \X_0
  \end{cases},
\end{equation}
for positive semi-definite matrices $Q$ and $R$.

We start by considering the dynamics around a nominal trajectory $(\X_l(t), u_l(t))$. Consider the error $\a_e = \X(t) \ominus_r \X_l(t)$. Since the error takes values in $T_{\X_l(t)} \M \cong \mathbb{R}^n$ the rule of total derivatives in Remark \ref{remark:total_derivative} applies and the error dynamics become
\begin{equation}
  \label{eq:mpc_linearized_dynamics}
  \begin{aligned}
    \frac{\mathrm{d} \a_e}{\mathrm{d}t} = \mathrm{d}^r (\a_e)_t
     & = \mathrm{d}^r (\X \ominus_r \X_l)_\X \mathrm{d}^r \X_t + \mathrm{d}^r (\X \ominus_r \X_l)_{\X_l} \; \mathrm{d}^r (\X_l)_t
    \\
     & \overset{\eqref{eq:d_rminus_fst}, \eqref{eq:d_rminus_snd}}= \left[\mathrm{d}^r \exp_{\a_e} \right]^{-1} f\left(\X_l \oplus_r \a_e, u_l + u_e \right) - \left[\mathrm{d}^l \exp _{\a_e} \right]^{-1} \; \mathrm{d}^r (\X_l)_t,
  \end{aligned}
\end{equation}
Thus we can change coordinates and rewrite \eqref{eq:mpc_ocp} as
\begin{equation}
  \label{eq:mpc_ocp_error}
  \begin{cases}
    \min        & \int_0^T \left\| \sqrt{Q(\tau)} \left( \left( \X_l(\tau) \oplus_r \a_e(\tau) \right) \ominus_r \X_d(\tau) \right)  \right\|_2^2 + \left\| \sqrt{R(\tau)} (u_l(\tau) + u_e(\tau) - u_d(\tau)) \right\|  \mathrm{d} \tau, \\

    \text{s.t.} & \eqref{eq:mpc_linearized_dynamics},                                                                                                                                                                                     \\
                & \a_e(0) = \X_0 \ominus \X_l(0).
  \end{cases},
\end{equation}
This is now a regular optimal control problem and we can proceed by linearizing around $(\a_e, u_e) = (0, 0)$ to obtain the linear time-varying system:
\begin{equation}
  \frac{\mathrm{d}}{\mathrm{d}t} \a_e = A(t) \a_e + B(t) u_e + E(t),
\end{equation}
where, since $\mathrm{d}^r \exp_0 = \mathrm{d}^l \exp_0 = I$,
\begin{align}
  A(t) & \coloneq \left. \frac{\mathrm{d}}{\mathrm{d}\a_e} \right|_{\a_e = 0} \left[\mathrm{d}^r \exp_{\a_e} \right]^{-1} f\left(\X_l(t) \oplus_r \a_e, u_l(t)  \right),  \\
  B(t) & \coloneq \left. \frac{\mathrm{d}}{\mathrm{d}u_e} \right|_{u_e = 0} f\left(\X_l(t), u_l(t) + u_e \right),  \\
  E(t) & \coloneq f\left(\X_l(t), u_l(t) \right) - \mathrm{d}^r(\X_l)_t.
\end{align}
To facilitate evaluating the const function we note that
\begin{equation}
  \left( \X_l \oplus_r \a_e \right) \ominus_r \X_d = \log \left( \X_d^{-1} \circ \X_l \circ \exp(\a_e) \right) = \log(\exp(\X_l \ominus_r \X_d) \circ \exp(\a_e)) \approx \X_l \ominus_r \X_d + \a_e(t),
\end{equation}
where the last approximate step follows from the Baker-Campbell-Hausdorff formula \eqref{eq:bch_formula}.

We can thus write it on the form
\begin{equation}
  \left\| \sqrt{Q} \left( \left( \X_l \oplus_r \a_e \right) \ominus_r \X_d \right)  \right\|_2^2 \approx \left( \X_l \ominus_r \X_d + \a_e \right)^T Q \left( \X_l \ominus_r \X_d + \a_e \right) = \a_e^T Q \a_e + 2\left( \X_l \ominus_r \X_d  \right)^T Q \a_e.
\end{equation}