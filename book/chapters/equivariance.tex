% !TEX root = ../root.tex

\chapter{Equivariance}

\todo[inline]{Write about equivariant systems}

\begin{itemize}
  \item Left-invariant, right-invariant, equivariant dynamical systems
\end{itemize}

Literature:
\begin{itemize}
  \item Grizzle on structure of control systems with symmetries \cite{grizzle_structure_1985}
  \item Modern paper on equivariant filtering \cite{van_goor_equivariant_2020}
\end{itemize}

One of the primary reasons to study Lie groups is to take advantage of the symmetry that comes with being a group. In this chapter we formalize the concept of symmetries and discuss how it can be leveraged.

Up until now we have taken $\M$ to be a Lie group, but in this chapter it denotes any manifold. On the other hand, $\G$ denotes a Lie group that \emph{acts} on the manifold. A special case is when $\M = \G$ and the action is that of group composition.

\begin{definition}
  Consider a manifold $\M$ and a Lie group $\G$. A function $\phi : \G \times \M \rightarrow \M$ is a (left) \textbf{action} if
  \begin{itemize}
    \item $\phi_\id(\x) = \x$ for all $\x \in \M$
    \item $\phi_h \circ \phi_g (\x)= \phi_{h \circ g} (\x)$ for all $h, g \in \G$ and all $\x \in \M$
  \end{itemize}
\end{definition}
When the actions are understood from the context we write $g . \x = \phi_g \x$ to denote the action of $g$ on $\x$. Some properties of actions are introduced next
\begin{definition}
  Consider an action $\phi$ of $\G$ on $\M$. It is said to be
  \begin{itemize}
    \item \textbf{Free} if $g . \x = \x$ if and only if $g = \id$,
    \item \textbf{Proper} if $(g, \x) \mapsto (\x, g.\x)$ is proper (the pre-image of compact sets is compact).
  \end{itemize}
\end{definition}
It is known that if an action is free and proper, then the quotient space
\begin{equation}
  M / G \coloneq \left\{ G x : \x \in \M \right\}
\end{equation}
is a smooth manifold \cite[Proposition 9.3.2]{marsden_introduction_1998}.

\begin{figure}
  \begin{center}
    \begin{tikzpicture}
      \node (m) {$\M$};
      \node[node distance=7em, right=of m] (n) {$\N$};
      \node[node distance=4em, below=of m] (m2) {$\M$};
      \node[node distance=4em, below=of n] (n2) {$\N$};
      \draw[-latex] (m) -- node[left] {$\phi^M$} (m2);
      \draw[-latex] (n) -- node[right] {$\phi^N$} (n2);
      \draw[-latex] (m) -- node[above] {$f$} (n);
      \draw[-latex] (m2) -- node[above] {$f$} (n2);
    \end{tikzpicture}
  \end{center}
  \caption{Commutation diagram of an equivariant function $f : \M \rightarrow \N$.}
  \label{fig:equivariance_diagram}
\end{figure}

We next define equivariance of a function with respect to two actions by the same Lie group.
\begin{definition}
  Let $\G$ be a Lie group that acts on two manifolds $\M$ and $\N$ via actions $\phi^\M$ and $\phi^\N$. Then a function $f : \M \rightarrow \N$ is \textbf{equivariant} with respect to $\left(\phi^\M, \phi^\N\right)$ if
  \begin{equation}
    f \circ \phi^M_g = \phi^N_g \circ f, \quad \forall g \in \G.
  \end{equation}
\end{definition}
Equivariance is illustrated by the commutation diagram in Figure \ref{fig:equivariance_diagram}. A special case of equivariance is \textbf{invariance}, which is when $\phi^N_g \x = \x$ is the trivial action.

Equivariance and invariance are useful since the behavior of an equivariant function $f : \M \rightarrow \N$ is completely characterized by the restricted function $f : \M / G \rightarrow \N$., where $\M / G$ is a lower-dimensional space.


\section{Equivariance in Dynamical Systems}

Equivariance can also be defined for dynamical systems.

\begin{figure}
  \begin{center}
    \begin{tikzpicture}
      \node (m) {$\M$};
      \node[node distance=7em, right=of m] (n) {$\lM$};
      \node[node distance=4em, below=of m] (m2) {$\M$};
      \node[node distance=4em, below=of n] (n2) {$\lM$};
      \draw[-latex] (m) -- node[left] {$\phi$} (m2);
      \draw[-latex] (n) -- node[right] {$\mathrm{d} \phi$} (n2);
      \draw[-latex] (m) -- node[above] {$f$} (n);
      \draw[-latex] (m2) -- node[above] {$f$} (n2);
    \end{tikzpicture}
  \end{center}
  \caption{Commutation diagram for an equivariant dynamical system.}
  \label{fig:system_equivariance_diagram}
\end{figure}

\begin{definition}
  Consider a dynamical system on a Lie group
  \begin{equation}
    \mathrm{d}^r \x_t = f(\x(t)).
  \end{equation}
  This system is said to be equivariant with respect to a $\G$-action $\phi_g$ on $\M$ if
  \begin{equation}
    f \circ \phi_g = (\mathrm{d}^r \phi_g) \circ f, \quad \forall g \in \G.
  \end{equation}
\end{definition}

\todo[inline]{Equivariant control system, prove closed-loop system equivariant}

We show how equivariance in a dynamical system can be applied in a Lyapunov context. Consider a function $h : M \rightarrow \mathbb{R}$ that is $G$-\textbf{invariant}, i.e. for all $g \in G$ it holds that
\begin{equation}
\label{eq:invariance}
  h \equiv h \circ \phi^g.
\end{equation}
We show that if $h$ is $G$-invariant, and $f$ is $G$-equivariant, then $\mathcal L_f h(X) \coloneq \langle \mathrm{d} h_{X}, f(X) \rangle$ is $G$-invariant as well. By differentiating \eqref{eq:invariance} at $X$ it follows that
\begin{equation}
  \mathrm{d} h_X = \mathrm{d} h_{\phi^g X} \circ \mathrm{d} \phi^g_{X} \quad \Leftrightarrow \quad \mathrm{d} h_{\phi^g X} = \mathrm{d} h_X \circ \left(\mathrm{d} \phi_X^g \right)^{-1}.
\end{equation}

Then indeed we have,
\begin{equation}
  \mathcal L_f h (\phi^g X) = \left \langle \mathrm{d} h_{\phi^g X}, (f \circ \phi^g) (X)\right \rangle = \left \langle \mathrm{d} h_X \circ (\mathrm{d} \phi_X^{g})^{-1} , (\mathrm{d} \phi^g_X \circ f)(X) \right \rangle = \langle \mathrm{d} h_X, f(X) \rangle,
\end{equation}
or in other words, $\mathcal L_f h$ is $G$-invariant:
\begin{equation}
  \mathcal L_f h \circ \phi^g  \equiv   \mathcal L_f h.
\end{equation}
Thus, if the dynamics are $G$-equivariant and $h$ is $G$-invariant, it is enough to assertain that a Lyapunov condition holds on a quotient space:
\begin{equation}
  \mathcal L_f h(X) \leq 0 \qquad \forall X \in M / G,
\end{equation}
since this implies that the property holds on all of $M$.


\section{Equivariance of Lie Group Control Systems}

If $M$ itself is a Lie Group we can write a second-order system on control-affine form as
\begin{equation}
\label{eq:lie_control_system}
  \begin{bmatrix} \mathrm{d} X_t \\ \mathrm{d} \omega_t \end{bmatrix} =  f\left( X, \omega, u \right) \coloneq \begin{bmatrix} \omega \\ e(X, \omega)u \end{bmatrix},
\end{equation}
where $f : M \times TM \times U \rightarrow TM \times TTM$. For a group $G$ acting on $M$ via $\phi^g : M \rightarrow M$ we can introduce a lifted action $\bar \phi_g : M \times TM \rightarrow M \times TM$ in a natural way by mapping the tangent element through the differential $\mathrm{d} \phi^g_X : T_X M \rightarrow T_{\phi^g X} M$ of $\phi^g$:
\begin{equation}
  \bar \phi^g \begin{bmatrix}
    X \\ \omega
  \end{bmatrix} = \begin{bmatrix} \phi^g X \\ \mathrm{d} \phi^g_X \omega \end{bmatrix}, \qquad \mathrm{d} \bar \phi^g_{X, \omega} \begin{bmatrix}
    \omega \\ a
  \end{bmatrix} = \begin{bmatrix} \mathrm{d} \phi^g_X \omega \\ \mathrm{d} \phi^g_X a \end{bmatrix},
\end{equation}
since $\mathrm{d} (\mathrm{d} \phi^g_X \omega)_\omega = \mathrm{d} \phi^g_X$. With this as the action the we can apply the equivariance ideas from above. We also introduce a separate action $\psi^g$ that represents the action of $g \in G$ on an element of the input space $U$. We say that the control system is $G$-equivariant if
\begin{equation}
  \label{eq:controlled_equivariance}
  f \left( \phi^g X, \mathrm{d} \phi^g_X \omega, \psi^g u \right) = \mathrm{d} \bar \phi_X^g f(X, \omega, u),
\end{equation}
which is illustrated as a commutation property in Figure \ref{fig:equivariance_diagram}.

\begin{proposition}
  Assume that the control system \eqref{eq:lie_control_system} satisfies the equi-variance property \eqref{eq:controlled_equivariance}. Then, if
  $u(X, \omega)$ is $G$-equivariant with respect to $(\bar \phi^g, \psi^g)$, i.e. 
  \begin{equation}
    \label{eq:input_equivariance}
    u(\phi^g X, \mathrm{d} \phi^g_X \omega) = \psi^g u(X, \omega),
  \end{equation}
  then the closed-loop system is also $G$-equivariant.
\end{proposition}
\begin{proof}
  From the \eqref{eq:controlled_equivariance} and \eqref{eq:input_equivariance} we have
  \begin{equation}
    f(\phi^g X, \mathrm{d} \phi^g_X \omega, u(\phi^g X, \mathrm{d} \phi^g_X \omega)) = f(\phi^g, \mathrm{d} \phi^g_X \omega, \psi^g u(X, \omega)) = \mathrm{d} \phi_X^g f(X, \omega, u(X, \omega)),
  \end{equation}
  which shows equivariance of the closed-loop system.
\end{proof}

