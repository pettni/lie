% !TEX root = ../manuscript.tex

\chapter{Lie Groups}
\label{chapter:lie_groups}

\begin{itemize_outcomes}
  \item Fundamental definitions and properties.
  \item Matrix Lie groups that appear in robotics.
\end{itemize_outcomes}


\section{Fundamentals}

A Lie group is an object that is both a group and a smooth manifold. As will be illustrated in these notes, inheritence of these two sets of properties places Lie groups at a unique point where theory meets practice.

We recall the definitions of groups and smooth manifolds, respectively.

\begin{definition}[\cite{fraleigh_first_2014}]
  \label{def:group}
  A \textbf{group} $(\M, \circ)$ is a set $\M$ closed under a binary operation ($\circ$) such that
  \begin{itemize}
    \item \textbf{associativity} holds: $\X \circ (\Y \circ \Z) = (\X \circ \Y) \circ \Z$ for all $\X,\Y,\Z \in \M$,
    \item there is an \textbf{identity element} $\id \in \M$ s.t. $\id \circ \X = \X \circ \id$ for all $\X \in \M$,
    \item for each element $\X \in \M$ there is an \textbf{inverse} $\X^{-1} \in \M$ s.t. $\X^{-1} \circ \X = \X \circ \X^{-1} = \id$.
  \end{itemize}
\end{definition}


\begin{definition}[\cite{do_carmo_riemannian_1992}]
  A \textbf{smooth manifold} $(\M, \{ \chart_i \})$ of dimension $n$ is a set $\M$ and a family of injective mappings $\chart_i : U_i \subset \mathbb{R}^n \rightarrow \M$ of open sets $U_i$ (called \textbf{charts}) such that
  \begin{enumerate}
    \item The charts cover the set: $\bigcup\limits_i U_i = \M$,
    \item For any pair $i, j$ with $\chart_i(U_i) \cap \chart_j(U_j) \eqcolon W \neq \emptyset$, the sets $\chart_i^{-1}(W)$ and $\chart_j^{-1}(W)$ are open in $\mathbb{R}^n$, and the mappings $\chart_i^{-1} \circ \chart_j$ are differentiable.
  \end{enumerate}
\end{definition}

\todo[inline]{Figure of chart mappings}

The definition of a Lie group is now straightforward.

\begin{important}
  \begin{definition}
    \label{def:lie_group}
    A \textbf{Lie group} of dimension $n$ is a set $\M$ together with a binary operation ($\circ$) and a family of injective mappings $\chart_i : U_i \subset \mathbb{R}^n \rightarrow X$ such that
    \begin{enumerate}
      \item $(\M, \circ)$ is a group,
      \item $(\M, \{ \chart_i \})$ is an $n$-dimensional smooth manifold.
    \end{enumerate}
  \end{definition}
\end{important}

In the following we use $\M$ to refer both to the Lie group and to its underlying set. A mathematic object that satisfies the properties in Definition \ref{def:lie_group} is the \textbf{general linear group} $GL(n, \mathbb{C})$---the set of $n \times n$ invertible complex matrices with matrix multiplication ($\cdot$) as the group operation. It turns out that many Lie groups of practical interest can be represented as sub-groups of $GL(n, \mathbb{C})$. In these notes we restrict attention to Matrix Lie groups.

\begin{definition}
  A \textbf{matrix Lie group} is a Lie group that is also a sub-group of $GL(n, \mathbb{C})$---the group of invertible matrices with complex coefficients.
\end{definition}

\begin{figure}
  \begin{center}
    \begin{tikzpicture}[
        setnode/.style = {
            draw,
            circle,
            minimum width=1.2cm
          },
        node distance=4em
      ]
      \node[setnode] (group) {$\cM$};
      \node[setnode, right=of group] (group_param) {$\M$};

      \draw[-latex] (group) to[bend left] node[above] {$\wedge$} (group_param);

      \draw[-latex] (group_param) to[bend left] node[above] {$\vee$} (group);
    \end{tikzpicture}
  \end{center}
  \caption{The $\vee$ (hat) and $\wedge$ (vee) maps map between the matrix and parameter forms of a matrix Lie group.}
\end{figure}


Lie theory is more straightforward to develop for matrix lie groups compared to a more general setting. Matrix Lie groups are however inefficient from a practical point of view since their representation is often redundant

We will distinguish between the matrix representation of a Lie group, which is useful for analytical purposes, and compact parameterized representations that are computationally more efficient. For a matrix Lie group $\M$ we denote the corresponding lower-dimensional representation $\cM$. The mappings $\vee$ and $\wedge$ are used to convert between the two representations and are smooth group homomorphisms, i.e.
\begin{equation}
  \x \circ \y = (\hat \x \cdot \hat \y)^{\vee}, \quad \x, \y \in \cM.
\end{equation}


\section{Examples of Lie Groups}

In the following we introduce the Matrix Lie group forms of various  Lie groups that are of practical interest in robotics. For every group we first introduce the matrix form $\M$ and present a lower-dimensional parameterization $\cM$ that is homomorphic to $\M$. We then perform matrix inverseion and multiplication in $\M$ to obtain the forms of inverse and composition also in $\cM$.

\subsection{\texorpdfstring{$\En$}{E(n)}: \texorpdfstring{$n$}{n}-dimensional translations}
We start with a group that is isometric to $\mathbb{R}^n$ under addition. For $\bp \in \mathbb{R}^n$ consider the matrix
\begin{equation}
  \label{eq:en_group}
  \bp^{\wedge} \coloneq \begin{bmatrix}
    I_n & \bp \\ \symbf{0}_{1 \times n} & 1
  \end{bmatrix}.
\end{equation}
When multiplying two such matrices the result is
\begin{equation}
  \begin{bmatrix}
    I_n & \bp \\ \symbf{0}_{1 \times n} & 1
  \end{bmatrix} \begin{bmatrix}
    I_n & \bp' \\ \symbf{0}_{1 \times n} & 1
  \end{bmatrix} = \begin{bmatrix}
    I_n & \bp + \bp' \\ \symbf{0}_{1 \times n} & 1
  \end{bmatrix},
\end{equation}
i.e. the result is still a matrix of the form in \eqref{eq:en_group}.

\paragraph{Lower-dimensional representation:} The Matrix lie group $\En$ consists of matrices of the form \eqref{eq:en_group}, but those matrices are parameterized by $n$ parameters, so $\check{\En} = \mathbb{R}^n$---the regular Euclidean vector space in $n$ dimensions.

\paragraph{Identity:} $\id_{\cEn} = \symbf{0}_n$ since $\symbf{0}_n^\wedge$ is the identity matrix.

\paragraph{Inverse:} Since $\begin{bmatrix}
    I_n & \bp \\ \symbf{0}_{1 \times n} & 1
  \end{bmatrix}^{-1} = \begin{bmatrix}
    I_n & -\bp \\ \symbf{0}_{1 \times n} & 1
  \end{bmatrix}$ the group inverse is negation: $\bp^{-1} = - \bp$.

\paragraph{Composition:} The embedding of $\bp \in \mathbb{R}^n$ in a matrix of the form \eqref{eq:en_group} is such that addition in $\mathbb{R}^n$ corresponds to matrix multiplication in the embedding, i.e.
\begin{equation}
  \bp^{\wedge} \cdot (\bp')^{\wedge} = (\bp + \bp')^{\wedge}.
\end{equation}
Therefore group composition is addition: $\bp \circ \bp' = \bp + \bp'$.

\paragraph{Action on $\mathbb{R}^n$:} We can define an action of $\En$ on $\mathbb{R}^n$ as translation. The action can be defined in terms of a matrix multiplication by associating $\bu \in \mathbb{R}^n$ with a homogenous vector ${\bu}^H = \begin{bmatrix}
    \bu \\ 1
  \end{bmatrix}$. Then
\begin{equation}
  \label{eq:en_action}
  \left \langle \bp^\wedge, {\bu^H }\right \rangle_{\En} = \begin{bmatrix}
    I_n & \bp \\ \symbf{0}_{1 \times n} & 1
  \end{bmatrix} \begin{bmatrix}
    \bu \\ 1
  \end{bmatrix} = \begin{bmatrix}
    \bu + \bp \\ 1
  \end{bmatrix} = (\bu + \bp)^H.
\end{equation}
This first example of a matrix Lie group is not very interesting in itself. However, by expressing Euclidean space $\mathbb{R}^n$ as a Lie group the fundamental strengths of Lie theory becomes apparent: the ability to treat any Lie group in the same way as the linear space $\mathbb{R}^n$.


\subsection{\texorpdfstring{$\SOtwo$}{SO(2)}: Two-dimensional rotations}

This Lie group consists of 2x2 real matrices with determinant equal to one, or, equivalently, orthogonal matrices with positive determinant.
\begin{equation}
  \SOtwo = \left\{ \bR = \begin{bmatrix}
    q_w & -q_z \\ q_z & q_w
  \end{bmatrix} \mid q_w^2 + q_z^2 = 1 \right\}.
\end{equation}

\paragraph{Lower-dimensional representation:} We use two parameters and one equality constraint to parameterize $\SOtwo$ .
\begin{equation}
  \cSOtwo = \left\{ (q_w, q_z) \in \mathbb{R}^2 \mid q_w^2 + q_z^2 = 1 \right\}.
\end{equation}

\paragraph{Identity:} The identity element $\id_{\cSOtwo} = (1, 0)$ correponds to the identity matrix $I_2 \in \SOtwo$.

\paragraph{Inverse:} The inverse is $(q_w, q_z)^{-1} = (q_w, -q_z)$ which corresponds to matrix transposition.

\paragraph{Composition:}
\begin{equation}
  \begin{aligned}
    (q_w, q_z) \circ (q_w', q_z') = \left( (q_w, q_z)^\wedge \cdot (q_w', q_z')^\wedge \right)^\vee                                    \\
    = \left( \begin{bmatrix}
      q_w & -q_z \\ q_z & q_w
    \end{bmatrix} \cdot \begin{bmatrix}
      q_w' & -q_z' \\ q_z' & q_w'
    \end{bmatrix}  \right)^\vee & = (q_w q_w' - q_z q_z', q_z q_w' + q_w q_z').
  \end{aligned}
\end{equation}

\paragraph{Action on $\mathbb{R}^2$:}
The group defines a rotation action on $\mathbb{R}^2$. For $\bR \in \SOtwo$ and $\bu \in \mathbb{R}^2$ the action is matrix multiplication
\begin{equation}
  \left \langle \bR, \bu \right \rangle_{\SOtwo} = \bR \cdot u.
\end{equation}


\subsection{\texorpdfstring{$\SEtwo$}{SE(2)}: Planar poses}

We now combine $\Etwo$ and $\SOtwo$ into a group that simultaneously represents translation and rotation in two dimensions. The result is $\SEtwo$---the special euclidean group in two dimensions.

In matrix form $\SEtwo$ consists of matrices on the form
\begin{equation}
  \label{eq:se2_matrix}
  \SOtwo = \left\{ \begin{bmatrix}
    \bR & \bp \\ \symbf{0}_{1 \times 2} & 1
  \end{bmatrix} \mid R \in \SOtwo \right\},
\end{equation}
i.e. the identity matrix block in $\eqref{eq:en_group}$ has been replaced with a member of $\SOtwo$. It follows that both $\SOtwo$ and $\Etwo$ are sub-groups of $\SEtwo$. In addition, $\Etwo$ is a normal subgroup\footnote{If $X \in \mathbb{E}^2$ and $Y \in \SOtwo$, then $Y X Y^{-1} \in \mathbb{E}^2$.} which implies that $\SEtwo$ is a \emph{semi-direct product} denoted $\SEtwo \cong \SOtwo \ltimes \Etwo$. Group products (direct and semi-direct) are discussed further below.

\paragraph{Lower-dimensional representation:} Four parameters are required, two for each subgroup
\begin{equation}
  \check{\mathbb{SE}}(2) = \cSOtwo  \times \cEtwo = \left\{ ((q_w, q_z), (p_x, p_y)) : (q_w, q_z) \in \cSOtwo , (p_x, p_y) \in \cEtwo \right\}.
\end{equation}

\paragraph{Identity:} The identity element is inherited from the sub-groups: $\id_{\check{\SEtwo}} = \left( \id_{\cSOtwo}, \id_{\check{\Etwo}} \right) = ((1, 0), (0, 0))$.

\paragraph{Inverse:} From matrix inverse it follows that $(\bR, \bp)^{-1} = (\bR^T, -\bR^T \bp)$.

\paragraph{Composition:} Matrix multiplication shows that composition in the lower-dimensional representation is
\begin{equation}
  (\bR, \bp) \circ (\bR', \bp') = (\bR \bR', \bR \bp' + \bp).
\end{equation}

\paragraph{Action on $\mathbb{\bR}^2$:}
This group has a natural action on two-dimensional vectors that consists of rotation and translation. For $X \coloneq (\bR, \bp) \in \check {\SEtwo}$ the action is
\begin{equation}
  \left \langle \X, \bu \right \rangle_{\SEtwo} = \left \langle \bR, \bu \right \rangle_{\SOtwo} + \bp = \bR \bu + \bp.
\end{equation}
That is, the vector $\bu$ is first rotated through the action of the $\SOtwo$ part of the state, and then subjected to a translation. Like in \eqref{eq:en_action} this action can be written as a matrix multiplication if we associate $\bu$ with its homogeneous counterpart $\bu^H$:
\begin{equation}
  \left \langle \X, \bu^H \right \rangle = \begin{bmatrix}
    \bR & \bp \\ \symbf{0}_{1 \times 2} & 1
  \end{bmatrix} \begin{bmatrix}
    \symbf u \\ 1
  \end{bmatrix} = \begin{bmatrix}
    \bR \bu + \bp \\ 1
  \end{bmatrix}.
\end{equation}
The action has a natural interpretation as a change of coordinates: if $\begin{bmatrix} \bR & \bp \\ \symbf{0}_{1 \times 2} & 1 \end{bmatrix} \in \SEtwo$, then $\left \langle \X, \bu \right \rangle$ represents the tranformation from a coordinate frame attached at $\bp$ with unit vectors the columns of $\bR$, to the global coordinate frame.


\subsection{Groups representing three-dimensional rotations}

As opposed to the 2D case where $\SOtwo$ as defined above is the canonical way to represent rotations, the situation is more complicated in three dimensions. While $\SOtwo$ generalizes to $\SOthree$ that consists of orthogonal $3 \times 3$ matrices with determinant 1, it is no longer as easy to construct a lower-dimensional representation. The usual choice is the unit quaternions, which are isomorphic to the matrix Lie group $\SUtwo$. We begin by defining the matrix group $\SOthree$.

\subsubsection{$\SOthree$: three-dimensional rotations}

$\SOthree$ is a matrix Lie group that consists of $3 \times 3$ orthogonal matrices with determinant equal to one:
\begin{equation}
  \SOthree = \left\{ \bR \in \mathbb{GL}(3) \mid \bR^T \bR = I, \det(\bR) = 1  \right\}.
\end{equation}
These matrices are usually referred to as \textbf{rotation matrices}.

There is no trivial low-dimensional parameterizations of this set, however, it is isometric to another group $\mathbb{SU}(2)$ that is in turn isometric to the unit quaternions $\Sthree$ which can be used as a lower-dimensional representation of $\SOthree$.
\begin{equation}
  \Sthree = \left \{ (q_w, q_x, q_y, q_z)  : q_w^2 + q_x^2 + q_y^2 + q_z^2 = 1 \right\}.
\end{equation}
However, the mapping is not 1-to-1, since both $\symbf{q} \coloneq (q_w, q_x, q_y, q_z)$ and $-\symbf{q}$ correspond to the same rotation matrix.


\paragraph{Action on $\mathbb{R}^3$}

The action of $\bR \in \SOthree$ on $\bu \in \mathbb{R}^3$ is rotation:
\begin{equation}
  \left \langle \bR, \bu \right \rangle = \bR \cdot \bu.
\end{equation}



\subsubsection{$\mathbb{SU}(2)$ and its relation to the quaternion group $\Sthree$}

We can associate a quaternion $\symbf{q} = q_w + q_x \symbf{i} + q_y \symbf{j} + q_z \symbf{k}$ with the unitary matrix
\begin{equation}
  \label{eq:su2_matrix}
  \mathbb{SU}(2) = \left\{ \begin{bmatrix}
    q_w + i q_z & -q_x - i q_y \\
    q_x - i q_y & q_w - i q_z
  \end{bmatrix} \mid q_w^2 + q_x^2 + q_y^2 + q_z^2 = 1 \right\}
\end{equation}
for which it holds that $A_{\quat_1 * \quat_2} = A_{\quat_1} A_{\quat_2}$. Thus the unit quaternions $\Sthree$ are isomorphic to $\mathbb{SU}(2)$ and can therefore be viewed as a matrix Lie group.

By muliplying two elements in $\mathbb{SU}$ we retrieve quaternion multiplication:
\begin{equation}
  \begin{aligned}
    \begin{bmatrix}
      q_w + i q_z & -q_x - i q_y \\
      q_x - i q_y & q_w - i q_z
    \end{bmatrix} \begin{bmatrix}
      q_w' + i q_z' & -q_x' - i q_y' \\
      q_x' - i q_y' & q_w' - i q_z'
    \end{bmatrix}
    = \begin{bmatrix}
      q_w'' + i q_z'' & -q_x'' - i q_y'' \\
      q_x'' - i q_y'' & q_w'' - i q_z''
    \end{bmatrix}
  \end{aligned}
\end{equation}
where
\begin{equation}
  \begin{aligned}
    q_w'' & = q_w q_w' - q_x q_x' - q_y q_y' - q_z q_z',  \\
    q_x'' & =  q_x q_w' + q_w q_x' + q_y q_z' - q_z q_y', \\
    q_y'' & = q_y q_w' + q_w q_y' + q_z q_x' - q_x q_z',  \\
    q_z'' & = q_z q_w' + q_w q_z' + q_x q_y' - q_y q_x',
  \end{aligned}
\end{equation}
which is exactly what is obtained by carrying out the usual quaternion multiplication
\begin{equation}
  (q_w + q_x \symbf{i} + q_y \symbf{j} + q_z \symbf{k}) * (q_w' + q_x' \symbf{i} + q_y' \symbf{j} + q_z' \symbf{k})
\end{equation}
with the quaternion rules $\symbf{i}\symbf{j} = \symbf{k}, \symbf{j}\symbf{k} = \symbf{i}, \symbf{k}\symbf{i} = \symbf{j}$ and $\symbf{i}^2 = \symbf{j}^2 = \symbf{k}^2 = -1$.

\paragraph{Action on $\mathbb{R}^3$}

A quaternion $\symbf{q} = q_w + q_x \symbf{i} + q_y \symbf{j} + q_z \symbf{k}$ acts on $\bu \coloneq \begin{bmatrix} u_x \\ u_y \\ u_z \end{bmatrix} \in \mathbb{R}^3$ as quaternion rotation $\symbf{q} * \bu * \bar {\symbf{q}}$ where $\bu$ is associated with the quaternion $u_x \symbf{i} + u_y \symbf{j} + u_z \symbf{k}$.

In terms of matrix multiplication operation can be written
\begin{equation}
  \begin{aligned}
    \begin{bmatrix}
      q_w + i q_z & -q_x - i q_y \\
      q_x - i q_y & q_w - i q_z
    \end{bmatrix} \begin{bmatrix}
      i u_z       & -u_x - i u_y \\
      u_x - i u_y & i u_z
    \end{bmatrix} \begin{bmatrix}
      q_w - i q_z   & q_x + i q_y \\
      - q_x + i q_y & q_w + i q_z
    \end{bmatrix} \\
    = \begin{bmatrix}
      i u_z'        & -u_x' - i u_y' \\
      u_x' - i u_y' & i u_z'
    \end{bmatrix}
  \end{aligned}
\end{equation}
for
\begin{equation}
  \begin{aligned}
    u_x' & = (1 - 2(q_y^2 + q_z^2)) u_x + 2(q_x q_y - q_w q_z) u_y + 2( q_x q_z + q_w q_y ) u_z   \\
    u_y' & = 2(q_x q_y + q_w q_z) u_x + (1 - 2 (q_x^2 + q_z^2)) u_y + 2( q_y q_z-q_w q_x) u_z     \\
    u_z' & = 2 (q_x q_z  - q_w q_y) u_x  + 2 (q_w q_x + q_y q_z) u_y + (1 - 2(q_x^2 + q_y^2)) u_z
  \end{aligned}
\end{equation}

Since this is a linear mapping in $u_x, u_y, u_z$ we can identify $\symbf{q}$ with a matrix $\bR(\bq)$ with coefficients
\begin{equation}
  \bR(\bq) = \begin{bmatrix}
    (1 - 2(q_y^2 + q_z^2)) & 2(q_x q_y - q_w q_z)    & 2( q_x q_z + q_w q_y ) \\
    2(q_x q_y + q_w q_z)   & (1 - 2 (q_x^2 + q_z^2)) & 2( q_y q_z-q_w q_x)    \\
    2 (q_x q_z  - q_w q_y) & 2 (q_w q_x + q_y q_z)   & (1 - 2(q_x^2 + q_y^2))
  \end{bmatrix}.
\end{equation}
Thus we can utilize the quaternion group $\Sthree$ as the lower-dimensional representation of $\SOthree$.

\begin{properties}[title=Useful quaternion identities]
  \paragraph{Axis-angle to quaternion} The quaternion $\bq$ representing the rotation about a unit axis $\symbf{\beta} = (\beta_x, \beta_y, \beta_z)$ for an angle $\alpha$ is
  \begin{equation}
    \label{eq:axis_angle_to_quaternion}
    \bq = \cos \left( \frac{\alpha}{2} \right) + \sin \left( \frac{\alpha}{2} \right) \left( \beta_x \symbf{i} + \beta_y \symbf{j} + \beta_z \symbf{k} \right).
  \end{equation}
  \paragraph{Two vectors to quaternion} A quaternion $\bq$ such that $\bq \bu = \bv$ for unit vectors $\bu, \bv$.
  \begin{equation}
    \label{eq:two_vectors_to_quaternion}
    \bq = \sqrt{\frac{1 + s}{2}} + \sqrt{\frac{1 - s}{2}} \left( \beta_x \symbf{i} + \beta_y \symbf{j} + \beta_z \symbf{k} \right), \quad s = \bu \cdot \bv, \; \symbf{\beta} = \bu \times \bv.
  \end{equation}
  \paragraph{Hopf fibration} The quaternions can be parameterized as the product of a rotation $\bq_\theta$ around the $z$ axis and a quaternion that rotates $\symbf{e}_z$ to $\symbf{\beta} \coloneq \begin{bmatrix} \beta_x, \beta_y, \beta_z \end{bmatrix} \in \mathbb{S}^2$ as
  \begin{equation}
    \bq = \bq_{\symbf{\beta}} * \bq_{\theta}, \quad \bq_{\symbf{\beta}} = \frac{1}{\sqrt{2(1 + \beta_z)}} \left( 1 + \beta_z - \symbf{i} \beta_x + \symbf{j} \beta_y \right), \; \; \bq_{\theta} = \cos \left( \frac{\theta}{2} \right) + \symbf{k} \sin \left( \frac{\theta}{2} \right).
  \end{equation}
  The special case when $\beta_z = -1$ is a singularity and must be handled separately, for example by setting $\bq_{\begin{bmatrix}0, 0, -1\end{bmatrix}} = \symbf{i}$. The Hopf parameterization is a manifestation of the fact that $\Sthree$ locally is a product of the spaces $\mathbb{S}^2$ and $\mathbb{S}^1$.
\end{properties}
\begin{proof}[Proof of \eqref{eq:two_vectors_to_quaternion}]
  From properties of the dot and cross products the sought-after rotation is about the axis $\symbf{\beta} = \bu \times \bv$ for the angle $\alpha$ such that $s \coloneq \bu \cdot \bv = \cos (\alpha)$. The half-angle formulas then give that $\cos (\alpha / 2) = \sqrt{(1 + s) / 2}$, and similarly for the sine part in \eqref{eq:axis_angle_to_quaternion}.
\end{proof}

\subsubsection{Summary of the three-dimensional rotation groups}

In summary, $\SOthree$ and $\SUtwo$ are both matrix Lie groups that represent rotations in three dimensions. $\SUtwo$ is isomorphic to the unit quaternions $\Sthree$ which is a good choice for the lower-dimensional representation. We also use the unit quaternions as the lower-dimensional representation of $\SOthree$ since a quaternion corresponds to a unique rotation matrix, and any rotation matrix corresponds to a unique (up to sign) quaternion.

\subsection{\texorpdfstring{$\SEthree$}{S(3)}: Three-dimensional poses}

Just as $\SEtwo$ was constructed as a semi-simple product of $\SOtwo$ and $\Etwo$, an analogous construction can be done in three dimensions. We define $\SEthree$ as as a semi-direct product $\SEthree \cong \SOthree \ltimes \mathbb{E}(3)$
\begin{equation}
  \label{eq:se3_matrix}
  \SEthree = \left\{ \begin{bmatrix} \bR & \bp \\ \symbf{0}_{1 \times 3} & 1 \end{bmatrix} \mid \bR \in \SOthree, \bp \in \symbf{E}(3) \right\}.
\end{equation}

The construction is analogous to the construction of $\SEthree$ above.


\section{Product Groups}

\todo[inline]{Discuss direct vs semi-direct products}

compare e.g. $\SEtwo \cong \SOtwo \ltimes \Etwo$ and the direct product $\SOtwo \otimes \Etwo$.



\section{Summary}

The different Lie groups introduced above are the following:

\begin{tabular}{lll}
  \toprule                                                                                                    \\
           & Matrix representation $\M$                       & Parameter representation $\cM$                \\
  \midrule                                                                                                    \\
  $E(n)$   & $n+1 \times n+1$ as in \eqref{eq:en_group}       & $\mathbb{R}^n$                                \\
  $\SOtwo$ & $2 \times 2$ orthogonal, determinant 1           & $(q_w, q_z) \in \check {\SOtwo}$              \\
  $SU(2)$  & $2 \times 2$ unitary                             & $(q_w, q_x, q_y, q_z) \in \Sthree$            \\
  $SO(3)$  & $3 \times 3$ orthogonal, determinant 1           & $(q_w, q_x, q_y, q_z) \in \Sthree$            \\
  $SE(2)$  & $SO(2) \ltimes E(2)$ as in \eqref{eq:se2_matrix} & $\check {\mathbb{SO}}(2) \times \mathbb{R}^2$ \\
  $SE(3)$  & $SE(3) \ltimes E(3)$ as in \eqref{eq:se3_matrix} & $\Sthree \times \mathbb{R}^3$                 \\
  \bottomrule
\end{tabular}
