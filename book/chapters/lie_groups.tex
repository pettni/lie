% !TEX root = ../root.tex

\chapter{Lie Groups}
\label{chapter:lie_groups}

\begin{itemize_outcomes}
  \item Fundamental definitions and properties.
  \item Matrix Lie groups that appear in robotics.
\end{itemize_outcomes}


\section{Fundamentals}

A Lie group is an object that is both a group and a smooth manifold. As will be illustrated in these notes, inheritence of these two sets of properties places Lie groups at a unique point where theory meets practice.

We recall the definitions of groups and smooth manifolds, respectively.

\begin{definition}[\cite{fraleigh_first_2014}]
  \label{def:group}
  A \textbf{group} $(\M, \circ)$ is a set $\M$ closed under a binary operation ($\circ$) such that
  \begin{itemize}
    \item \textbf{associativity} holds: $\X \circ (\Y \circ \Z) = (\X \circ \Y) \circ \Z$ for all $\X,\Y,\Z \in \M$,
    \item there is an \textbf{identity element} $\id \in \M$ s.t. $\id \circ \X = \X \circ \id$ for all $\X \in \M$,
    \item for each element $\X \in \M$ there is an \textbf{inverse} $\X^{-1} \in \M$ s.t. $\X^{-1} \circ \X = \X \circ \X^{-1} = \id$.
  \end{itemize}
\end{definition}


\begin{definition}[\cite{do_carmo_riemannian_1992}]
  A \textbf{smooth manifold} $(\M, \{ \chart_i \})$ of dimension $n$ is a set $\M$ and a family of injective mappings $\chart_i : U_i \subset \mathbb{R}^n \rightarrow \M$ of open sets $U_i$ (called \textbf{charts}) such that
  \begin{enumerate}
    \item The charts cover the set: $\bigcup\limits_i U_i = \M$,
    \item For any pair $i, j$ with $\chart_i(U_i) \cap \chart_j(U_j) \eqcolon W \neq \emptyset$, the sets $\chart_i^{-1}(W)$ and $\chart_j^{-1}(W)$ are open in $\mathbb{R}^n$, and the mappings $\chart_i^{-1} \circ \chart_j$ are differentiable.
  \end{enumerate}
\end{definition}

\todo[inline]{Figure of chart mappings}

The definition of a Lie group is now straightforward.

\begin{important}
  \begin{definition}
    \label{def:lie_group}
    A \textbf{Lie group} of dimension $n$ is a set $\M$ together with a binary operation ($\circ$) and a family of injective mappings $\chart_i : U_i \subset \mathbb{R}^n \rightarrow X$ such that
    \begin{enumerate}
      \item $(\M, \circ)$ is a group,
      \item $(\M, \{ \chart_i \})$ is an $n$-dimensional smooth manifold.
    \end{enumerate}
  \end{definition}
\end{important}

In the following we use $\M$ to refer both to the Lie group and to its underlying set. A mathematic object that satisfies the properties in Definition \ref{def:lie_group} is the \textbf{general linear group} $GL(n, \mathbb{C})$---the set of $n \times n$ invertible complex matrices with matrix multiplication ($\cdot$) as the group operation. It turns out that many Lie groups of practical interest can be represented as sub-groups of $GL(n, \mathbb{C})$. In these notes we restrict attention to Matrix Lie groups.

\begin{definition}
  A \textbf{matrix Lie group} is a Lie group that is also a sub-group of $GL(n, \mathbb{C})$---the group of invertible matrices with complex coefficients.
\end{definition}

\begin{figure}
  \begin{center}
    \begin{tikzpicture}[
        setnode/.style = {
            draw,
            circle,
            minimum width=1.2cm
          },
        node distance=4em
      ]
      \node[setnode] (group) {$\cM$};
      \node[setnode, right=of group] (group_param) {$\M$};

      \draw[-latex] (group) to[bend left] node[above] {$\wedge$} (group_param);

      \draw[-latex] (group_param) to[bend left] node[above] {$\vee$} (group);
    \end{tikzpicture}
  \end{center}
  \caption{The $\vee$ (hat) and $\wedge$ (vee) maps map between the matrix and parameter forms of a matrix Lie group.}
\end{figure}

Lie theory is more straightforward to develop for matrix lie groups compared to a more general setting. Matrix Lie groups are however inefficient from a practical point of view since their representation is often redundant. For this first part of the book we focus exclusively on matrix Lie groups. In Part II we use isometries between matrix Lie groups and more concice non-matrix Lie groups to obtain closed-form formulas for the latter.
