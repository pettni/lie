% !TEX root = ../root.tex

\chapter{\texorpdfstring{$\SEthree$}{SE(3)}: The 3D Rigid Motion Group}

Moving to three dimensions changes little---the properties of $\SEthree$ are quite similar to those of $\SEtwo$. It is formed as a semi-direct product between $\SOthree$ and $\Tthree$
\begin{equation}
  \SEthree = \SOthree \ltimes \Tthree = \left\{ \X = \begin{bmatrix}
    \bR & \bp \\ 0 & 1
  \end{bmatrix} \mid \bR \in \SOthree, \bp \in \mathbb{R}^3 \right\}.
\end{equation}
A rotation plus translation action on $\bu \in \mathbb{R}^3$ is defined by
\begin{equation}
  \left \langle \X, \bu \right \rangle_\SEthree = \left \langle \bR, \bu \right \rangle_\SOthree + \bp = \bR \bu + \bp.
\end{equation}
The Lie algebra has six degrees of freedom that we parameterize by
\begin{center}
  \begin{tikzpicture}
    \node (a1) {$\mathbb{R}^{6} \ni \begin{bmatrix}  \bv \\ \bomega \end{bmatrix}$};
    \node at (5, 0) (a2) {$\begin{bmatrix} \bomega^\wedge & \bv \\ 0 & 0 \end{bmatrix} \in \sethree$};
    \draw[-latex] (a1) to[bend left] node[above] {$\wedge$} (a2.north west);
    \draw[-latex] (a2.south west) to[bend left] node[above] {$\vee$} (a1);
  \end{tikzpicture}
\end{center}
where $\bv, \bomega \in \mathbb{R}^3$ are the linear and angular components, respectively.

\section{Formulas}

\paragraph{Adjoint} From the definition \eqref{eq:def_bad},
\begin{equation*}
  \begin{aligned}
    \bAd_\X \begin{bmatrix}
      \bv \\ \bomega
    \end{bmatrix} & = \left( \begin{bmatrix}
      \bR & \bp \\ 0 & 1
    \end{bmatrix} \begin{bmatrix}
      \hat \bomega & \bv \\ 0 & 0
    \end{bmatrix}\begin{bmatrix}
      \bR & \bp \\ 0 & 1
    \end{bmatrix}^{-1} \right)^\vee
    = \begin{bmatrix}
      \bR \bomega^\wedge \bR^T & -\bR \bomega^\wedge \bR^T \bp + \bR \bv \\ 0 & 1
    \end{bmatrix}^\vee                                                                                                            \\
                                      & = \begin{bmatrix} \bR \bv - \bR \bomega^\wedge \bR^T \bp \\ (\bR \bomega^\wedge \bR)^\vee \end{bmatrix}
    \overset{\eqref{eq:so3_Ad}}{=} \begin{bmatrix} \bR \bv - (\bR \bomega)^\wedge \bp \\ \bR \bomega \end{bmatrix} \overset{\eqref{eq:so3_hat_neg}}{=} \begin{bmatrix} \bR \bv + \bp^\wedge \bR \bomega \\ \bR \bomega \end{bmatrix} = \begin{bmatrix} \bR & \bp^\wedge \bR \\ 0 & \bR \end{bmatrix} \begin{bmatrix} \bv \\ \bomega \end{bmatrix},
  \end{aligned}
\end{equation*}
thus the uppercase adjoint is
\begin{equation}
  \bAd_\X = \begin{bmatrix}
    \bR & \bp^\wedge \bR \\ 0 & \bR
  \end{bmatrix}.
\end{equation}

\paragraph{Exponential and Logarithm}

Looking to again utilize Lemma \ref{lem:help_exp} we calculate
\begin{equation}
  S(\bomega) \coloneq \sum\limits_{k=0}^\infty \frac{\hat{\bomega}^{k}} {(k+1)!} = I_3 - \frac{1}{\| \symbf\omega \|^2} {\symbf{\symbf}} \sum\limits_{k=2}^\infty \frac{\hat{\bomega}^k}{k!} = I_3  - \frac{1}{\| \symbf\omega \|^2} \hat {\bomega} \left( \Exp_\SOthree(\hat {\bomega}) - I_3 - \hat{\bomega} \right).
\end{equation}
From the $\SOthree$ exponential in \eqref{eq:so3_exp},
\begin{equation}
  \label{eq:so3_leftjac}
  \begin{aligned}
    S(\bomega)
     & = I_3  - \frac{1}{\| \bomega \|^2} \hat {\bomega} \left( \left( I_3 + \frac{ \sin \| \bomega \|}{\| \bomega \|} \hat{\bomega} + \frac{ (1 - \cos \| \bomega \|)}{\| \bomega \|^2} \hat{\bomega}^2 \right) - I_3 - \hat{\bomega} \right)
    \\
     & = I_3 - \frac{ (\sin \| \bomega \| -\| \bomega \|)  }{\| \bomega\|^3} \hat {\bomega}^2 + \frac{\| \bomega \|^2  (1 - \cos \| \bomega \| ) }{\| \bomega \|^4} \hat {\bomega}
    \\
     & = I_3 + \frac{ \| \bomega \| - \sin \| \bomega \| }{\| \bomega\|^3} \hat {\bomega}^2 + \frac{1 - \cos \| \bomega \|}{\| \bomega \|^2} \hat {\bomega} .
  \end{aligned}
\end{equation}
Remark that for $\SEthree$ it happens that $S(\bomega) = \dl \left( \exp_\SOthree \right)_\bomega$, which is a difference from $\SEtwo$ where there is no direct relation between $S(\omega_z)$ and the derivatives of the exponential on $\SOtwo$.

Applying Lemma \ref{lem:help_exp} then gives the sought-after $\SEthree$ exponential
\begin{equation}
  \exp_\SEthree (\bomega, \bv) = \Exp_\SEthree \begin{bmatrix}
    \hat{\bomega} & \symbf{v} \\ \symbf{0}_{1 \times 3} & 0
  \end{bmatrix} = \begin{bmatrix}
    \exp_{\SOthree}(\bomega) & S(\bomega) \symbf v \\ \symbf{0}_{1 \times 3} & 1
  \end{bmatrix},
\end{equation}
and consequently the logarithm
\begin{equation}
  \log_\SEthree \begin{bmatrix}
    \bR & \bp \\ \symbf{0}_{1 \times 3} & 1
  \end{bmatrix}                                = \left(\symbf \alpha, S(\symbf \alpha)^{-1} \bp  \right),
\end{equation}
where $\symbf \alpha = \log_\SOthree R$.

\paragraph{Derivatives of the Exponential}

First we derive an expression for $\ad_\a$ utilizing that for the hat operator on $\SOthree$,  $\hat \a \b = -\hat \b \a$
\begin{equation}
  \begin{aligned}
    \left[ \begin{bmatrix}
        \bv \\ \bomega
      \end{bmatrix}, \begin{bmatrix}
        \bar \bv \\ \bar \bomega
      \end{bmatrix} \right]_\SEthree = \left(\begin{bmatrix} \hat \bomega & \bv \\ 0 & 0 \end{bmatrix}\begin{bmatrix} \hat {\bar \bomega} & \bar{\bv} \\ 0 & 0 \end{bmatrix} - \begin{bmatrix} \hat {\bar \bomega} & \bar{\bv} \\ 0 & 0 \end{bmatrix} \begin{bmatrix} \hat \bomega & \bv \\ 0 & 0 \end{bmatrix} \right)^\vee= \begin{bmatrix} \left[ \bomega, \bar \bomega \right]^\wedge_\SOthree & \hat \bomega \bar \bv - \hat {\bar \bomega} \bv \\ 0 & 0 \end{bmatrix}^\vee \\
    = \begin{bmatrix}  \hat \bomega \bar \bv - \hat {\bar \bomega} \bv \\ \left[ \bomega, \bar \bomega \right]_\SOthree  \end{bmatrix} = \underbrace{\begin{bmatrix}
        \hat \bomega & \hat \bv \\ 0 & \hat \bomega
      \end{bmatrix}}_{\ad_\a} \begin{bmatrix}
      \bar v \\ \bar \bomega
    \end{bmatrix}.
  \end{aligned}
\end{equation}
We are interested in the powers $\ad_\a^k$ in order to evaluate the exponential derivatives. For $k \geq 1$
\begin{equation}
  \ad_\a^k = \begin{bmatrix}
    \hat \bomega & \hat \bv \\ 0 & \hat \bomega
  \end{bmatrix}^k
  = \begin{bmatrix}
    \hat \bomega^k & \sum_{i=0}^{k-1} \hat \bomega^{i} \hat \bv \hat \bomega^{k-1-i} \\ 0 & \hat \bomega^k
  \end{bmatrix}.
\end{equation}
Thus the left derivative of the exponential can be written
\begin{equation*}
  \mathrm{d}^l \exp_\a = \sum_{k = 0}^\infty \frac{\ad_\a^k}{(k+1)!} = I + \sum_{k = 1}^\infty \frac{1}{(k+1)!}  \begin{bmatrix}
    \hat \bomega^k & \sum_{i = 0}^{k-1} \hat \bomega^i \hat \bv \hat \bomega^{k - 1 - i} \\ 0 & \hat \bomega^k
  \end{bmatrix} = \begin{bmatrix}
    \mathrm{d}^l \left(\exp_{\SOthree}\right)_\bomega & Q^l(\bv, \bomega)                                 \\
    0                                                 & \mathrm{d}^l \left(\exp_{\SOthree}\right)_\bomega
  \end{bmatrix},
\end{equation*}
where a closed-form expression for $Q^l(\bv, \bomega)$ can be painstakingly obtained through a series of sum manipulations. We first convert the formula to a form that is symmetric in $i$ and $k$.
\begin{equation*}
  \begin{aligned}
    Q^l(\bv, \bomega) & \coloneq \sum_{k = 1}^\infty \frac{1}{(k+1)!} \sum_{i=0}^{k-1} \hat \bomega^i \hat \bv \hat \bomega^{k - 1 - i} = \sum_{k = 0}^\infty \sum_{i=0}^{k} \frac{1}{(k+2)!} \hat \bomega^i \hat \bv \hat \bomega^{k - i} \\
                      & = \sum_{i = 0}^\infty \sum_{k = i}^\infty \frac{1}{(k+2)!} \hat \bomega^i \hat \bv \hat \bomega^{k - i} = \sum_{i = 0}^\infty \sum_{k = 0}^\infty \frac{1}{(k+i+2)!} \hat \bomega^i \hat \bv \hat \bomega^{k}.
  \end{aligned}
\end{equation*}
With the same steps the right derivative can be shown to be
\begin{equation}
  \label{eq:se3_qr}
  Q^r(\bv, \bomega) \coloneq \sum_{k=1}^\infty \frac{(-1)^k}{(k+1)!} \sum_{i=0}^{k-1} \hat \bomega ^i \hat \bv \hat \bomega^{k-1-i} = \ldots = - \sum_{i = 0}^\infty \sum_{k = 0}^\infty \frac{(-1)^{k+i}}{(k+i+2)!} \hat \bomega^i \hat \bv \hat \bomega^{k}
\end{equation}
and we can see that $Q^r$ is conveniently obtained from $Q^l$ by
\begin{equation}
  Q^r(\bv, \bomega) = Q^l(-\bv, -\bomega)
\end{equation}
which is convenient to know since calculating one of them is tedious enough.

In the following calculation the sum $\sum_{k, i \geq 0}$ is first split into parts $(k=i=0)$, $(k=0, i \geq 1)$, $(k \geq 1, i = 0)$ and $(k, i \geq 1)$, and then the resulting single sums are split into two sums $i = 0, 2, \ldots$ and $i = 1, 3, \ldots$. Also using that
\begin{equation}
  \label{eq:hat_parity_rels}
  \hat \bomega^{2k+1} = (-1)^{k} \| \bomega \|^{2k}\hat \bomega, \qquad \hat \bomega^{2k+2} = (-1)^{k} \| \bomega \|^{2k}\hat \bomega^2,
\end{equation}
which follows from \eqref{eq:so3_pow3}, we get
\begin{equation*}
  \begin{aligned}
     & Q^l (\bv, \bomega) = \frac{1}{2} \hat \bv + \sum_{i = 1}^\infty \frac{\hat \bomega^i \hat \bv}{(i+2)!} + \sum_{k = 1}^\infty \frac{\hat \bv\hat \bomega^k }{(k+2)!} + \sum_{i=1}^\infty \sum_{k=1}^\infty \frac{1}{(i+k+2)!} \hat \bomega^i \hat \bv \hat \bomega^k                                                                                                                            \\
     & = \frac{1}{2} \hat \bv + \sum_{i = 0}^\infty \frac{\hat \bomega^{i+1} \hat \bv + \hat \bv \hat \bomega^{i+1}}{(i+3)!} + \sum_{i=0}^\infty \sum_{k=0}^\infty \frac{1}{(i+k+4)!} \hat \bomega^{i+1} \hat \bv \hat \bomega^{k+1}                                                                                                                                                                  \\
     & = \frac{1}{2} \hat \bv + \sum_{i=0}^\infty \frac{\hat \bomega^{2i+1} \hat \bv + \hat \bv \hat \bomega^{2i+1}}{(2i+3 )!} + \sum_{i=0}^\infty \frac{\hat \bomega^{2i+2} \hat \bv + \hat \bv \hat \bomega^{2i+2}}{(2i+4)!} + \sum_{i=0}^\infty \sum_{k=0}^\infty \frac{1}{(i+k+4)!} \hat \bomega^{i+1} \hat \bv \hat \bomega^{k+1}                                                                \\
     & = \frac{1}{2} \hat \bv + \sum_{i=0}^\infty \frac{(-1)^{i}\| \bomega \|^{2i}}{(2i+3)!}  \left( \hat \bomega \hat \bv + \hat \bv \hat \bomega \right)  + \sum_{i=0}^\infty \frac{(-1)^{i}\| \bomega \|^{2i}}{(2i+4)!}  \left( \hat \bomega^2 \hat \bv + \hat \bv \hat \bomega^2 \right) + \sum_{i=0}^\infty \sum_{k=0}^\infty \frac{1}{(i+k+4)!} \hat \bomega^{i+1} \hat \bv \hat \bomega^{k+1}.
  \end{aligned}
\end{equation*}
The first two sums are given in \eqref{eq:trig_sum3} and \eqref{eq:trig_sum4}:
\begin{equation}
  \label{eq:se3_sine_sums}
  \sum_{i=0}^\infty \frac{(-1)^i}{(2i+3)!} \| \bomega \|^{2i}   = \frac{\| \bomega \| - \sin \| \bomega \|}{\| \bomega \|^3}, \quad
  \sum_{i=0}^\infty \frac{(-1)^{i}}{(2i+4)!} \| \bomega \|^{2i} =  \frac{\cos \| \bomega \| - 1 + \frac{\|\bomega\|^2}{2}}{\| \bomega \|^4}.
\end{equation}
The double sum requries additional work. Using $\hat \bomega \hat \bv \hat \bomega = (-\bomega \cdot \bv) \hat \bomega$ from \eqref{eq:so3_pow3} yields
\begin{equation*}
  \sum_{i = 0}^\infty \sum_{k = 0}^\infty \frac{1}{(k+i+4)!} \hat \bomega^{i+1} \hat \bv \hat \bomega^{k+1} = (-\bomega \cdot \bv) \sum_{i = 0}^\infty \sum_{k = 0}^\infty \frac{1}{(k+i+4)!} \hat \bomega^{k+i+1}  \overset{j = k+i}=  (-\bomega \cdot \bv) \sum_{j = 0}^\infty \sum_{k = 0}^{j} \frac{1}{(j+4)!} \hat \bomega^{j+1},
\end{equation*}
and we can evaluate the sum as
\begin{equation*}
  \begin{aligned}
    \sum_{j=0}^\infty \sum_{k=0}^j \frac{1}{(j+4)!} \hat \bomega^{j+1} = - \sum_{j = 0}^\infty \frac{j+1}{(j+4)!} \hat \bomega^{j+1} = - \sum_{j=0}^\infty \left(\frac{1}{(j+3)!} - \frac{3}{(j+4)!} \right) \hat \bomega^{j+1} \\
    = - \sum_{j=0}^\infty \left(\frac{1}{(2j+3)!} - \frac{3}{(2j+4)!} \right) \hat \bomega^{2j+1} + \sum_{j=0}^\infty \left(\frac{1}{(2j+4)!} - \frac{3}{(2j+5)!} \right) \hat \bomega^{2j+2}                                   \\
    \overset{\eqref{eq:hat_parity_rels}}=   -\sum_{j=0}^\infty \left(\frac{(-1)^j  }{(2j+3)!}\| \bomega \|^{2j} + 3\frac{(-1)^j }{(2j+4)!} \| \bomega \|^{2j}\right)  \hat \bomega + \sum_{j=0}^\infty \left(-\frac{(-1)^j  }{(2j+4)!} \| \bomega \|^{2j} + 3\frac{(-1)^j  }{(2j+5)!}\| \bomega \|^{2j} \right) \hat \bomega^2 .
  \end{aligned}
\end{equation*}
The sums in \eqref{eq:se3_sine_sums} appear again and can be re-used, and the remaining sum with denominator $(2j+5)!$ was given in \eqref{eq:trig_sum5}:
\begin{equation}
  \sum_{j=0}^\infty \frac{(-1)^j}{(2j+5)!} \| \bomega \|^{2j} = \frac{\sin \|\bomega \| - \| \bomega \| + \frac{\|\bomega\|}{6}}{\| \bomega \|^5}.
\end{equation}
\begin{important}
  After collecting the various expressions the closed-form expression for $Q^l$ can be written down
  \begin{equation}
    \label{eq:se3_q_expr}
    \begin{aligned}
      Q^l (\bv, \bomega) & = \frac{1}{2}\hat \bv + \frac{\| \bomega \| - \sin \| \bomega \|}{\| \bomega \|^3}\left(\hat \bomega \hat \bv + \hat \bv \hat \bomega - (\bomega \cdot \bv) \hat \bomega \right)                     \\
                         & + \frac{ \cos \| \bomega \| - 1 + \frac{\|\bomega\|^2}{2} }{\| \bomega \|^4} \left(\hat \bomega^2 \hat \bv + \hat \bv \hat \bomega^2 + (\bomega \cdot \bv) (3 \hat \bomega - \hat \bomega^2) \right) \\
                         & - 3(\bomega \cdot \bv) \left(  \frac{\|\bomega\| - \sin \|\bomega\| - \frac{\|\bomega\|^3}{6}}{\|\bomega\|^5} \right) \hat \bomega^2 .
    \end{aligned}
  \end{equation}
\end{important}
The $Q$ matrix allows us to write down a closed-form expression for $\mathrm{d}^l \exp_\a$ on $\SEthree$, and $\left( \mathrm{d}^l \exp_\a\right)^{-1}$ follows from noting that $\begin{bmatrix} \A & \B \\ 0 & \A \end{bmatrix}^{-1} = \begin{bmatrix} \A^{-1} & -\A^{-1} \B \A^{-1} \\ 0 & \A^{-1} \end{bmatrix}$ for $\A$ invertible.

\begin{properties}[breakable, title={$\SEthree$ formula sheet}]
  Consists of $3 \times 3$ matrices $\X = \begin{bmatrix} \bR & \bp \\ 0 & 1 \end{bmatrix}$ that act on $\mathbb{R}^3$ via $\bu \mapsto \bR \bu + \bp$.

  \paragraph{Algebra Parameterization}
  \begin{equation}
    \left\{ \begin{bmatrix}
      \bv \\ \bomega
    \end{bmatrix} \mid \bv \in \mathbb{R}^3, \bomega \in ??? \right\}, \quad \begin{bmatrix}
      \bv \\ \bomega
    \end{bmatrix}^\wedge = \begin{bmatrix}
      \bomega^\wedge & \bv \\ 0 & 0
    \end{bmatrix} \in \sethree.
  \end{equation}

  \paragraph{Adjoint}
  \begin{equation}
    \bAd_X = \begin{bmatrix}
      \bR & \bp^\wedge \bR \\ 0 & \bR
    \end{bmatrix}.
  \end{equation}

  \paragraph{Exponential and Logarithm}
  \begin{subequations}
    \label{eq:se3_exp_log}
    \begin{align}
      \label{eq:se3_exp}
      \exp_\SEthree \begin{bmatrix}
        \bv \\ \bomega
      \end{bmatrix} & = \begin{bmatrix}
        \exp_{\SOthree}(\bomega) & S(\bomega) \symbf v \\ \symbf{0}_{1 \times 3} & 1
      \end{bmatrix}, \\
      \label{eq:se3_log}
      \log_\SEthree \begin{bmatrix}
        \bR & \bp \\ \symbf{0}_{1 \times 3} & 1
      \end{bmatrix} & = \begin{bmatrix}
        S(\symbf \alpha)^{-1} \bp \\
        \symbf \alpha
      \end{bmatrix},
    \end{align}
  \end{subequations}
  where $\symbf \alpha = \log_\SOthree(\bomega)$ and $S(\symbf \alpha) = \dl \left( \exp_\SOthree \right)_{\symbf \alpha}$.

  \paragraph{Bracket and Lowercase adjoint}
  \begin{align}
    \left[ \begin{bmatrix}
        \bv \\ \bomega
      \end{bmatrix}, \begin{bmatrix} \bv' \\ \bomega' \end{bmatrix}  \right] & = \begin{bmatrix} \bomega^\wedge \bv' - (\bomega')^\wedge \bv \\ \left[ \bomega, \bomega' \right]_\SOthree \end{bmatrix}   \\
    \ad_{\a}                                                               & =  \begin{bmatrix}  \hat \bomega & \hat \bv \\ 0 & \hat \bomega \end{bmatrix}.
  \end{align}
  Note that in these formulas $\bomega^\wedge$ and $\bv^\wedge$ denote the hat operator on $\SOthree$.

  \paragraph{Derivatives of the Exponential}
  Let $Q^{l/r}$ be as in \eqref{eq:se3_q_expr},
  \begin{subequations}
    \begin{align}
      \mathrm{d}^r \exp_{\a}                     & =   \begin{bmatrix} J^r_\SOthree & Q^l(-\bv, - \bomega) \\ 0 & J^r_\SOthree \end{bmatrix}, \\
      \mathrm{d}^l \exp_{\a}                     & = \begin{bmatrix} J^l_\SOthree & Q^l(\bv, \bomega) \\ 0 & J^l_\SOthree\end{bmatrix}    \\
      \left( \mathrm{d}^r \exp_{\a} \right)^{-1} & =  \begin{bmatrix} \left(J^r_\SOthree\right)^{-1}  &  -\left(J^r_\SOthree\right)^{-1} Q^l(-\bv, -\bomega)  \left(J^r_\SOthree\right)^{-1} \\ 0 &  \left(J^r_\SOthree\right)^{-1} \end{bmatrix},
      \\
      \left( \mathrm{d}^l \exp_{\a} \right)^{-1} & = \begin{bmatrix} \left(J^l_\SOthree\right)^{-1}  & -\left(J^l_\SOthree\right)^{-1} Q^l(\bv, \bomega)  \left(J^l_\SOthree\right)^{-1} \\ 0 &  \left(J^l_\SOthree\right)^{-1} \end{bmatrix} .
    \end{align}
  \end{subequations}
  where $J^{l/r}_\SOthree = \mathrm{d}^{l/r} \left( \exp_{\SOthree} \right)_\bomega$ and $\left(J^{l/r}_\SOthree \right)^{-1} = \left( \mathrm{d}^{l/r} \left( \exp_{\SOthree} \right)_\bomega \right)^{-1}$.
\end{properties}

\section{Parameterization via Isomorphism with \texorpdfstring{$\Sthree \ltimes \mathbb{R}^{3}$}{S3 |x R3}}

Similarly to $\SEtwo \cong \Uone \ltimes \mathbb{R}^2$ we utilize that $\SEthree \cong \Sthree \times \mathbb{R}^3$ to get a compact representation of $\SEthree$ elements and with group operation
\begin{equation}
  \label{eq:s3_r3_composition}
  (\bq, \bp) \circ (\bq, \bp') = \left( \bq \circ \bq', \left \langle \bq, \bp' \right \rangle_\Sthree + \bp \right).
\end{equation}

The hat and vee maps between $\SEthree$ and $\Sthree \ltimes \mathbb{R}^3$ are
\begin{center}
  \begin{tikzpicture}
    \node (a1) {$\Sthree \ltimes \mathbb{R}^3 \ni (\bq, \bp)$};
    \node at (5, 0) (a2) {$\begin{bmatrix} \bq^\wedge & \bp \\ 0 & 1\end{bmatrix} \in \SEthree$};
    \draw[-latex] (a1.north east) to[bend left] node[above] {$\wedge$} (a2.north west);
    \draw[-latex] (a2.south west) to[bend left] node[above] {$\vee$} (a1.south east);
  \end{tikzpicture}
\end{center}
from where the exponential and log maps follow from \eqref{eq:se3_exp_log}.

\begin{properties}[breakable, title={$\Sthree \ltimes \mathbb{R}^3$ formula sheet}]
  \paragraph{Group Definition}
  The parameterization of $\Sthree \ltimes \mathbb{R}^3$ is
  \begin{equation}
    \left\{ (\bq, \bp) \in \Sthree \times \mathbb{R}^3 \right\}
  \end{equation}
  and the group operation is defined in \eqref{eq:s3_r3_composition}. $\Sthree \ltimes \mathbb{R}^3$ is isomorphic to $\SEthree$ and inherits its Lie algebra properties.

  \paragraph{Frame Transformation Action on $\bu \in \mathbb{R}^{3}$}
  \begin{equation}
    \left\langle (\bq, \bp), \bu \right \rangle_{\Sthree \ltimes \mathbb{R}^3} = \left \langle \bq, \bu \right \rangle_\Sthree + \bp.
  \end{equation}

  \paragraph{Exponential and Logarithm}
  \begin{subequations}
    \label{eq:s3r3_exp_log}
    \begin{align}
      \label{eq:s3r3_exp}
      \exp \begin{bmatrix} \bv \\ \bomega \end{bmatrix} & = \left(\exp_\Sthree(\bomega), S(\bomega) \bv\right),      \\
      \label{eq:s3r3_log}
      \log (\bq, \bp)                   & = \begin{bmatrix} S(\symbf \alpha)^{-1} \bp \\ \symbf \alpha \end{bmatrix}, \quad \symbf \alpha = \log_\Sthree \bq.
    \end{align}
  \end{subequations}
  where $\symbf \alpha = \log_\SOthree(\bomega)$ and $S(\symbf \alpha) = \dl \left( \exp_\SOthree \right)_{\symbf \alpha}$.
\end{properties}

