\chapter{\texorpdfstring{$\SEthree$}{SE(3)}: The 3D Rigid Motion Group}

\paragraph{Parameterization}

\paragraph{Action on $\mathbb{R}^3$:}

\paragraph{Lie Algebra}

\section{Formulas}

\paragraph{Adjoint}

\paragraph{Exponential and Logarithm}

Continuing with $\SEthree$ we utilize \eqref{eq:so3_pow3} to calculate
\begin{equation}
  \sum\limits_{k=0}^\infty \frac{\hat{\bomega}^{k}} {(k+1)!} = I_3 - \frac{1}{\| \symbf\omega \|^2} \hat {\symbf{\symbf}} \sum\limits_{k=2}^\infty \frac{\hat{\bomega}^k}{k!} = I_3  - \frac{1}{\| \symbf\omega \|^2} \hat {\bomega} \left( \Exp_\SOthree(\hat {\bomega}) - I_3 - \hat{\bomega} \right).
\end{equation}
From \eqref{eq:so3_exp} we then have that
\begin{equation}
  \label{eq:so3_leftjac}
  \begin{aligned}
    \mathrm{d}^l \left( \exp_{\SOthree} \right)_\bomega
     & \coloneq \sum\limits_{k=0}^\infty \frac{\hat{\bomega}^{k}} {(k+1)!} = I_3  - \frac{1}{\| \bomega \|^2} \hat {\bomega} \left( \left( I_3 + \frac{ \sin \| \bomega \|}{\| \bomega \|} \hat{\bomega} + \frac{ (1 - \cos \| \bomega \|)}{\| \bomega \|^2} \hat{\bomega}^2 \right) - I_3 - \hat{\bomega} \right)
    \\
     & = I_3 - \frac{ (\sin \| \bomega \| -\| \bomega \|)  }{\| \bomega\|^3} \hat {\bomega}^2 + \frac{\| \bomega \|^2  (1 - \cos \| \bomega \| ) }{\| \bomega \|^4} \hat {\bomega}
    \\
     & = I_3 + \frac{ \| \bomega \| - \sin \| \bomega \| }{\| \bomega\|^3} \hat {\bomega}^2 + \frac{1 - \cos \| \bomega \|}{\| \bomega \|^2} \hat {\bomega} .
  \end{aligned}
\end{equation}
Applying Lemma \ref{lem:help_exp} then gives the exponential.
\begin{important}
  The matrix exponential on $\mathfrak{se}(3)$ is
  \begin{subequations}
    \begin{align}
      \exp_\SEthree (\bomega, \bv) = \Exp_\SEthree \begin{bmatrix}
        \hat{\bomega} & \symbf{v} \\ \symbf{0}_{1 \times 3} & 0
      \end{bmatrix} & = \begin{bmatrix}
        \Exp_{\SOthree}(\hat{\bomega}) & \mathrm{d}^l \left( \exp_{\SOthree} \right)_\bomega \symbf v \\ \symbf{0}_{1 \times 3} & 1
      \end{bmatrix},                                                                                        \\
      \Log_\SEthree \begin{bmatrix}
        \bR & \bp \\ \symbf{0}_{1 \times 3} & 1
      \end{bmatrix}                                & = \begin{bmatrix}
        \hat{\symbf \alpha} & \left( \mathrm{d}^l \left( \exp_{\SOthree} \right)_{\symbf \alpha} \right)^{-1} \bp \\ \symbf{0}_{1 \times 3} & 0
      \end{bmatrix},                                                                                        \\
      \log_\SEthree \begin{bmatrix}
        \bR & \bp \\ \symbf{0}_{1 \times 3} & 1
      \end{bmatrix}                                & = \left(\symbf \alpha, \left( \mathrm{d}^l \left( \exp_{\SOthree} \right)_{\symbf \alpha} \right)^{-1} \bp  \right),
    \end{align}
  \end{subequations}
  where $\symbf \alpha = \log_\SOthree R$.
\end{important}

\paragraph{Derivatives of the Exponential}

First we derive an expression for $\ad_\a$ utilizing that for the hat operator on $\SOthree$,  $\hat \a \b = -\hat \b \a$
\begin{equation}
  \begin{aligned}
    \left[ \begin{bmatrix}
        \bv \\ \bomega
      \end{bmatrix}, \begin{bmatrix}
        \bar \bv \\ \bar \bomega
      \end{bmatrix} \right]_\SEthree = \left(\begin{bmatrix} \hat \bomega & \bv \\ 0 & 0 \end{bmatrix}\begin{bmatrix} \hat {\bar \bomega} & \bar{\bv} \\ 0 & 0 \end{bmatrix} - \begin{bmatrix} \hat {\bar \bomega} & \bar{\bv} \\ 0 & 0 \end{bmatrix} \begin{bmatrix} \hat \bomega & \bv \\ 0 & 0 \end{bmatrix} \right)^\vee= \begin{bmatrix} \left[ \bomega, \bar \bomega \right]^\wedge_\SOthree & \hat \bomega \bar \bv - \hat {\bar \bomega} \bv \\ 0 & 0 \end{bmatrix}^\vee \\
    = \begin{bmatrix}  \hat \bomega \bar \bv - \hat {\bar \bomega} \bv \\ \left[ \bomega, \bar \bomega \right]_\SOthree  \end{bmatrix} = \underbrace{\begin{bmatrix}
        \hat \bomega & \hat \bv \\ 0 & \hat \bomega
      \end{bmatrix}}_{\ad_\a} \begin{bmatrix}
      \bar v \\ \bar \bomega
    \end{bmatrix}.
  \end{aligned}
\end{equation}
We are interested in the powers $\ad_\a^k$ in order to evaluate the exponential derivatives. For $k \geq 1$
\begin{equation}
  \ad_\a^k = \begin{bmatrix}
    \hat \bomega & \hat \bv \\ 0 & \hat \bomega
  \end{bmatrix}^k
  = \begin{bmatrix}
    \hat \bomega^k & \sum_{i=0}^{k-1} \hat \bomega^{i} \hat \bv \hat \bomega^{k-1-i} \\ 0 & \hat \bomega^k
  \end{bmatrix}.
\end{equation}

Thus the left derivative of the exponential can be written
\begin{equation*}
  \mathrm{d}^l \exp_\a = \sum_{k = 0}^\infty \frac{\ad_\a^k}{(k+1)!} = I + \sum_{k = 1}^\infty \frac{1}{(k+1)!}  \begin{bmatrix}
    \hat \bomega^k & \sum_{i = 0}^{k-1} \hat \bomega^i \hat \bv \hat \bomega^{k - 1 - i} \\ 0 & \hat \bomega^k
  \end{bmatrix} = \begin{bmatrix}
    \mathrm{d}^l \left(\exp_{\SOthree}\right)_\bomega & Q^l(\bv, \bomega)                                 \\
    0                                                 & \mathrm{d}^l \left(\exp_{\SOthree}\right)_\bomega
  \end{bmatrix},
\end{equation*}
where a closed-form expression for $Q^l(\bv, \bomega)$ can be painstakingly obtained through a series of sum manipulations. We first convert the formula to a form that is symmetric in $i$ and $k$.
\begin{equation*}
  \begin{aligned}
    Q^l(\bv, \bomega) & \coloneq \sum_{k = 1}^\infty \frac{1}{(k+1)!} \sum_{i=0}^{k-1} \hat \bomega^i \hat \bv \hat \bomega^{k - 1 - i} = \sum_{k = 0}^\infty \sum_{i=0}^{k} \frac{1}{(k+2)!} \hat \bomega^i \hat \bv \hat \bomega^{k - i} \\
                      & = \sum_{i = 0}^\infty \sum_{k = i}^\infty \frac{1}{(k+2)!} \hat \bomega^i \hat \bv \hat \bomega^{k - i} = \sum_{i = 0}^\infty \sum_{k = 0}^\infty \frac{1}{(k+i+2)!} \hat \bomega^i \hat \bv \hat \bomega^{k}.
  \end{aligned}
\end{equation*}
With the same steps the right derivative can be shown to be
\begin{equation}
  \label{eq:se3_qr}
  Q^r(\bv, \bomega) \coloneq \sum_{k=1}^\infty \frac{(-1)^k}{(k+1)!} \sum_{i=0}^{k-1} \hat \bomega ^i \hat \bv \hat \bomega^{k-1-i} = \ldots = - \sum_{i = 0}^\infty \sum_{k = 0}^\infty \frac{(-1)^{k+i}}{(k+i+2)!} \hat \bomega^i \hat \bv \hat \bomega^{k}
\end{equation}
and we can see that
\begin{equation}
  Q^r(\bv, \bomega) = Q^l(-\bv, -\bomega)
\end{equation}
which is convenient to know since calculating one of them is tedious enough.

In the following calculation the sum $\sum_{k, i \geq 0}$ is first split into parts $(k=i=0)$, $(k=0, i \geq 1)$, $(k \geq 1, i = 0)$ and $(k, i \geq 1)$, and then the resulting single sums are split into two sums $i = 0, 2, \ldots$ and $i = 1, 3, \ldots$. Also using that
\begin{equation}
  \label{eq:hat_parity_rels}
  \hat \bomega^{2k+1} = (-1)^{k} \| \bomega \|^{2k}\hat \bomega, \qquad \hat \bomega^{2k+2} = (-1)^{k} \| \bomega \|^{2k}\hat \bomega^2,
\end{equation}
which follows from \eqref{eq:so3_pow3}, we get
\begin{equation*}
  \begin{aligned}
     & Q^l (\bv, \bomega) = \frac{1}{2} \hat \bv + \sum_{i = 1}^\infty \frac{\hat \bomega^i \hat \bv}{(i+2)!} + \sum_{k = 1}^\infty \frac{\hat \bv\hat \bomega^k }{(k+2)!} + \sum_{i=1}^\infty \sum_{k=1}^\infty \frac{1}{(i+k+2)!} \hat \bomega^i \hat \bv \hat \bomega^k                                                                                                                            \\
     & = \frac{1}{2} \hat \bv + \sum_{i = 0}^\infty \frac{\hat \bomega^{i+1} \hat \bv + \hat \bv \hat \bomega^{i+1}}{(i+3)!} + \sum_{i=0}^\infty \sum_{k=0}^\infty \frac{1}{(i+k+4)!} \hat \bomega^{i+1} \hat \bv \hat \bomega^{k+1}                                                                                                                                                                  \\
     & = \frac{1}{2} \hat \bv + \sum_{i=0}^\infty \frac{\hat \bomega^{2i+1} \hat \bv + \hat \bv \hat \bomega^{2i+1}}{(2i+3 )!} + \sum_{i=0}^\infty \frac{\hat \bomega^{2i+2} \hat \bv + \hat \bv \hat \bomega^{2i+2}}{(2i+4)!} + \sum_{i=0}^\infty \sum_{k=0}^\infty \frac{1}{(i+k+4)!} \hat \bomega^{i+1} \hat \bv \hat \bomega^{k+1}                                                                \\
     & = \frac{1}{2} \hat \bv + \sum_{i=0}^\infty \frac{(-1)^{i}\| \bomega \|^{2i}}{(2i+3)!}  \left( \hat \bomega \hat \bv + \hat \bv \hat \bomega \right)  + \sum_{i=0}^\infty \frac{(-1)^{i}\| \bomega \|^{2i}}{(2i+4)!}  \left( \hat \bomega^2 \hat \bv + \hat \bv \hat \bomega^2 \right) + \sum_{i=0}^\infty \sum_{k=0}^\infty \frac{1}{(i+k+4)!} \hat \bomega^{i+1} \hat \bv \hat \bomega^{k+1}.
  \end{aligned}
\end{equation*}
The first two sums can now be evaluated in a fairly straightforward manner:
\begin{subequations}
  \begin{align}
    \label{eq:sine_sum}
    \sum_{i=0}^\infty \frac{(-1)^i}{(2i+3)!} \| \bomega \|^{2i}   & = - \frac{1}{\| \bomega \|^3} \sum_{i=0}^\infty \frac{(-1)^{i+1}}{(2(i+1)+1)!} \| \bomega \|^{2(i+1) + 1} = \frac{\| \bomega \| - \sin \| \bomega \|}{\| \bomega \|^3} ,         \\
    \label{eq:cosine_sum}
    \sum_{i=0}^\infty \frac{(-1)^{i}}{(2i+4)!} \| \bomega \|^{2i} & = \frac{1}{\| \bomega \|^4} \sum_{i=0}^\infty \frac{(-1)^{i+2}}{(2(i+2))!} \| \bomega \|^{2(i + 2)} =  \frac{\cos \| \bomega \| - 1 + \frac{\|\bomega\|^2}{2}}{\| \bomega \|^4}.
  \end{align}
\end{subequations}
The double sum requries additional work. Using $\hat \bomega \hat \bv \hat \bomega = (-\bomega \cdot \bv) \hat \bomega$ from \eqref{eq:so3_pow3} yields
\begin{equation*}
  \begin{aligned}
    \sum_{i = 0}^\infty \sum_{k = 0}^\infty \frac{1}{(k+i+4)!} \hat \bomega^{i+1} \hat \bv \hat \bomega^{k+1} = (-\bomega \cdot \bv) \sum_{i = 0}^\infty \sum_{k = 0}^\infty \frac{1}{(k+i+4)!} \hat \bomega^{k+i+1}  \overset{j = k+i}=  (-\bomega \cdot \bv) \sum_{j = 0}^\infty \sum_{k = 0}^{j} \frac{1}{(j+4)!} \hat \bomega^{j+1} \\
    = -(\bomega \cdot \bv) \sum_{j = 0}^\infty \frac{j+1}{(j+4)!} \hat \bomega^{j+1} = -(\bomega \cdot \bv) \sum_{j=0}^\infty \left(\frac{1}{(j+3)!} - \frac{3}{(j+4)!} \right) \hat \bomega^{j+1}                                                                                                                                      \\
    = -(\bomega \cdot \bv) \left( \sum_{j=0}^\infty \left(\frac{1}{(2j+3)!} - \frac{3}{(2j+4)!} \right) \hat \bomega^{2j+1} + \sum_{j=0}^\infty \left(\frac{1}{(2j+4)!} - \frac{3}{(2j+5)!} \right) \hat \bomega^{2j+2} \right)                                                                                                         \\
    \overset{\eqref{eq:hat_parity_rels}}= (\bomega \cdot \bv) \left( -\sum_{j=0}^\infty \left(\frac{(-1)^j  }{(2j+3)!}\| \bomega \|^{2j} + 3\frac{(-1)^j }{(2j+4)!} \| \bomega \|^{2j}\right)  \hat \bomega + \sum_{j=0}^\infty \left(-\frac{(-1)^j  }{(2j+4)!} \| \bomega \|^{2j} + 3\frac{(-1)^j  }{(2j+5)!}\| \bomega \|^{2j} \right) \hat \bomega^2 \right).
  \end{aligned}
\end{equation*}
The sums \eqref{eq:sine_sum} and \eqref{eq:cosine_sum} appear again and can be re-used. The remaining sum with denominator $(2j+5)!$ evaluates to a higher-order sine expression as follows
\begin{equation}
  \sum_{j=0}^\infty \frac{(-1)^j}{(2j+5)!} \| \bomega \|^{2j} = \frac{1}{\|\bomega\|^5} \sum_{j=0}^\infty \frac{(-1)^{j+2}}{((2j+2)+1)} \| \bomega \|^{2(j+2)+1} = \frac{\sin \|\bomega \| - \| \bomega \| + \frac{\|\bomega\|}{6}}{\| \bomega \|^5}.
\end{equation}
\begin{important}
  After collecting the various expressions the closed-form expression for $Q^l$ can be written down
  \begin{equation}
    \label{eq:se3_q_expr}
    \begin{aligned}
      Q^l (\bv, \bomega) & = \frac{1}{2}\hat \bv + \frac{\| \bomega \| - \sin \| \bomega \|}{\| \bomega \|^3}\left(\hat \bomega \hat \bv + \hat \bv \hat \bomega - (\bomega \cdot \bv) \hat \bomega \right)                     \\
                         & + \frac{ \cos \| \bomega \| - 1 + \frac{\|\bomega\|^2}{2} }{\| \bomega \|^4} \left(\hat \bomega^2 \hat \bv + \hat \bv \hat \bomega^2 + (\bomega \cdot \bv) (3 \hat \bomega - \hat \bomega^2) \right) \\
                         & - 3(\bomega \cdot \bv) \left(  \frac{\|\bomega\| - \sin \|\bomega\| - \frac{\|\bomega\|^3}{6}}{\|\bomega\|^5} \right) \hat \bomega^2 .
    \end{aligned}
  \end{equation}
\end{important}
The $Q$ matrix allows us to write down a closed-form expression for $\mathrm{d}^l \exp_\a$ on $\SEthree$, and $\left( \mathrm{d}^l \exp_\a\right)^{-1}$ follows from noting that $\begin{bmatrix} A & B \\ 0 & A \end{bmatrix}^{-1} = \begin{bmatrix} A^{-1} & -A^{-1} B A^{-1} \\ 0 & A^{-1} \end{bmatrix}$ for $A$ invertible.

\begin{properties}[title={Lowercase adjoint and exponential derivatives on $\SEthree$}]
  Let $\a = \begin{bmatrix} \bv \\ \bomega \end{bmatrix}$ and $Q^l$ as in \eqref{eq:se3_q_expr}. Then
  \begin{align}
    \ad_{\a}                                   & =  \begin{bmatrix}  \hat \bomega & \hat \bv \\ 0 & \hat \bomega \end{bmatrix},  \\
    \mathrm{d}^r \exp_{\a}                     & =   \begin{bmatrix} J^r_\SOthree & Q^l(-\bv, - \bomega) \\ 0 & J^r_\SOthree \end{bmatrix}, \\
    \mathrm{d}^l \exp_{\a}                     & = \begin{bmatrix} J^l_\SOthree & Q^l(\bv, \bomega) \\ 0 & J^l_\SOthree\end{bmatrix}    \\
    \left( \mathrm{d}^r \exp_{\a} \right)^{-1} & =  \begin{bmatrix} \left(J^r_\SOthree\right)^{-1}  &  -\left(J^r_\SOthree\right)^{-1} Q^l(-\bv, -\bomega)  \left(J^r_\SOthree\right)^{-1} \\ 0 &  \left(J^r_\SOthree\right)^{-1} \end{bmatrix},
    \\
    \left( \mathrm{d}^l \exp_{\a} \right)^{-1} & = \begin{bmatrix} \left(J^l_\SOthree\right)^{-1}  & -\left(J^l_\SOthree\right)^{-1} Q^l(\bv, \bomega)  \left(J^l_\SOthree\right)^{-1} \\ 0 &  \left(J^l_\SOthree\right)^{-1} \end{bmatrix} .
  \end{align}
  where $J^{l/r}_\SOthree = \left( \mathrm{d}^{l/r} \exp_{\SOthree} \right)_\bomega$ and $\left(J^{l/r}_\SOthree \right)^{-1} = \left( \left( \mathrm{d}^{l/r} \exp_{\SOthree} \right)_\bomega \right)^{-1}$. Note that in these formulas $\hat \bomega$ and $\hat \bv$ denote the hat operator on $\SOthree$.
\end{properties}

\begin{properties}[breakable, title={Formulas for $\SEthree$}]
\paragraph{Group Parameterization}
\begin{equation}
\end{equation}

\paragraph{Algebra Parameterization}
\begin{equation}
\end{equation}

\paragraph{Group Operations}
\begin{itemize}
  \item Identity element:
  \item Inverse:
  \item Composition:
\end{itemize}

\paragraph{Adjoint}
\begin{equation}
\end{equation}

\paragraph{Exponential}
\begin{equation}
\end{equation}

\paragraph{Logarithm}
\begin{equation}
\end{equation}

\paragraph{Lowercase adjoint}
\begin{equation}
\end{equation}

\paragraph{Derivatives of the Exponential}
\begin{equation}
\end{equation}
\end{properties}

\section{Parameterization via Isomorphism with \texorpdfstring{$\Sthree \ltimes \mathbb{R}^{3}$}{S3 |x R3}}

%%% Local Variables:
%%% mode: latex
%%% TeX-master: "../root"
%%% End:
